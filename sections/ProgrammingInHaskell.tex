%!TEX root = ../AdvPrPa.tex
\section{Programming in Haskell}
\begin{multicols}{2}
\subsection{First Steps}
\subsubsection{List functions}
\begin{tabular}{|p{0.5\linewidth}|p{0.5\linewidth}|}
  \hline
  \textbf{input} & \textbf{output} \\
  \hline
  \lstinline|head [1,2,3,4,5]| & \lstinline|1| \\
  \hline
  \lstinline|tail [1,2,3,4,5]| & \lstinline|[2,3,4,5]| \\
  \hline
  \lstinline|[1,2,3,4,5] !! 2| & \lstinline|3| \\
  \hline
  \lstinline|take 3 [1,2,3,4,5]| & \lstinline|[1,2,3]| \\
  \hline
  \lstinline|drop 3 [1,2,3,4,5]| & \lstinline|[4,5]| \\
  \hline
  \lstinline|length [1,2,3,4,5]| & \lstinline|5| \\
  \hline
  \lstinline|sum [1,2,3,4,5]| & \lstinline|15| \\
  \hline
  \lstinline|product [1,2,3,4,5]| & \lstinline|120| \\
  \hline
  \lstinline|[1,2,3] ++ [4,5]| & \lstinline|[1,2,3,4,5]| \\
  \hline
  \lstinline|reverse [1,2,3,4,5]| & \lstinline|[5,4,3,2,1]| \\
  \hline
\end{tabular}

\subsubsection{Function Application}
In mathematics, function application is denoted using parentheses, and multiplication is often denoted using juxtaposition or space:\\
$f(a,b) + c d$\\
In Haskell, function application is denoted using space, and multiplication is denoted using $*$:\\
\lstinline{f a b + c*d}\\
Moreover, function application is assumed to have higher priority than all other operators:\\
\begin{tabular}{|p{0.5\linewidth}|p{0.5\linewidth}|}
    \hline
    \lstinline|f a + b| & \lstinline|(f a) + b|, \textbf{not} \lstinline|f(a + b)|\\
    \hline
\end{tabular}

Examples:\\
\begin{tabular}{|p{0.5\linewidth}|p{0.5\linewidth}|}
\hline
\textbf{Mathematics} & \textbf{Haskell} \\
\hline
$f(x)$ & \lstinline|f x| \\
$f(x,y)$ & \lstinline|f x y| \\
$f(g(x))$ & \lstinline|f (g x)| \\
$f(x,g(y))$ & \lstinline|f x (g y)| \\
$f(x)g(y)$ & \lstinline|f x * g y| \\
\hline
\end{tabular}

\subsubsection{Haskell Scripts}
\begin{itemize}
    \item As well as the functions in the standard library, you can also define your own functions
    \item New functions are defined within a script, a text file comprising a sequence of definitions
    \item By convention, Haskell scripts usually have a .hs suffix on their filename. This is not mandatory, but is useful for identification purposes.
\end{itemize}

\subsubsection{My First Script}
\lstinputlisting{listings/test.hs}
Note:
\begin{itemize}
    \item \lstinline{div} is enclosed in back quotes, not forward
    \item \lstinline{x 'f' y} is just syntactic sugar for \lstinline{f x y}.
\end{itemize}

To start up GHCi with the script, type the following in a terminal:
\begin{lstlisting}[language=bash]
$ ghci test.hs
\end{lstlisting}
Now both the standard library and the file test.hs are loaded, and functions from both can be used:
\begin{lstlisting}[language=bash]
> quadruple 10
40
> take (double 2) [1,2,3,4,5,6]
[1,2,3,4]
\end{lstlisting}
GHCi does not automatically detect that the script has been changed, so a \lstinline[language=bash]{reload} command must be executed before the new definitions can be used:
\begin{lstlisting}[language=bash]
> :reload
Reading file "test.hs"
\end{lstlisting}

\subsubsection{Useful GHCi Commands}
\begin{tabular}{|p{0.5\linewidth}|p{0.5\linewidth}|}
  \hline
  \textbf{Command} & \textbf{Meaning} \\
  \hline
  \lstinline[language=bash]|:load name| & load script \lstinline[language=bash]|name| \\
  \lstinline[language=bash]|:reload| & reload current script \\
  \lstinline[language=bash]|:set editor name| & set editor to \lstinline[language=bash]|name| \\
  \lstinline[language=bash]|:edit name| & edit script \lstinline[language=bash]|name| \\
  \lstinline[language=bash]|:edit| & edit current script \\
  \lstinline[language=bash]|:type expr| & show type of \lstinline[language=bash]|expr| \\
  \lstinline[language=bash]|:?| & show all commands \\
  \lstinline[language=bash]|:quit| & quit GHCi \\
  \hline
\end{tabular}

\subsubsection{Naming Requirements}
\begin{itemize}
  \item Function and argument names must begin with a lower-case letter:\\
  \lstinline{myFun}, \lstinline{fun1}, \lstinline{arg_2}, \lstinline{x'}
  \item By convention, list arguments usually have an s suffix on their name:\\
  \lstinline{xs}, \lstinline{ns}, \lstinline{nss}
\end{itemize}

\subsubsection{The Layout Rule}
In a sequence of definitions, each definition must begin in precisely the same column:
\begin{multicols}{3}
correct:
\begin{lstlisting}
a = 10
b = 20
c = 30
\end{lstlisting}
\vfill\null
\columnbreak
wrong:
\begin{lstlisting}
a = 10
 b = 20
c = 30
\end{lstlisting}
\vfill\null
\columnbreak
wrong:
\begin{lstlisting}
 a = 10
b = 20
 c = 30
\end{lstlisting}
\vfill\null
\end{multicols}
The layout rule avoids the need for explicit syntax to indicate the grouping of definitions.
\begin{multicols}{2}
implicit grouping:
\begin{lstlisting}
a = b + c
    where
      b = 1
      c = 2
d = a * 2
\end{lstlisting}
\vfill\null
\columnbreak
explicit grouping:
\begin{lstlisting}
a = b + c
    where
    {b = 1;
     c = 2}
d = a * 2
\end{lstlisting}
\vfill\null
\end{multicols}



\end{multicols}
