%!TEX root = ../AdvPrPa.tex
\section{Programming in Haskell}
\begin{multicols}{2}
\subsection{First Steps}
\subsubsection{List functions}
\begin{tabularx}{\linewidth}{|X|X|}
  \hline
  \textbf{input} & \textbf{output} \\
  \hline
  \lstinline|head [1,2,3,4,5]| & \lstinline|1| \\
  \hline
  \lstinline|tail [1,2,3,4,5]| & \lstinline|[2,3,4,5]| \\
  \hline
  \lstinline|[1,2,3,4,5] !! 2| & \lstinline|3| \\
  \hline
  \lstinline|take 3 [1,2,3,4,5]| & \lstinline|[1,2,3]| \\
  \hline
  \lstinline|drop 3 [1,2,3,4,5]| & \lstinline|[4,5]| \\
  \hline
  \lstinline|length [1,2,3,4,5]| & \lstinline|5| \\
  \hline
  \lstinline|sum [1,2,3,4,5]| & \lstinline|15| \\
  \hline
  \lstinline|product [1,2,3,4,5]| & \lstinline|120| \\
  \hline
  \lstinline|[1,2,3] ++ [4,5]| & \lstinline|[1,2,3,4,5]| \\
  \hline
  \lstinline|reverse [1,2,3,4,5]| & \lstinline|[5,4,3,2,1]| \\
  \hline
\end{tabularx}

\subsubsection{Function Application}
In mathematics, function application is denoted using parentheses, and multiplication is often denoted using juxtaposition or space:\\
$f(a,b) + c d$\\
In Haskell, function application is denoted using space, and multiplication is denoted using $*$:\\
\lstinline{f a b + c*d}\\
Moreover, function application is assumed to have higher priority than all other operators:\\
\begin{tabularx}{\linewidth}{|X|X|}
  \hline
  \lstinline|f a + b| & \lstinline|(f a) + b|, \textbf{not} \lstinline|f(a + b)|\\
  \hline
\end{tabularx}

Examples:\\
\begin{tabularx}{\linewidth}{|X|X|}
\hline
\textbf{Mathematics} & \textbf{Haskell} \\
\hline
$f(x)$ & \lstinline|f x| \\
$f(x,y)$ & \lstinline|f x y| \\
$f(g(x))$ & \lstinline|f (g x)| \\
$f(x,g(y))$ & \lstinline|f x (g y)| \\
$f(x)g(y)$ & \lstinline|f x * g y| \\
\hline
\end{tabularx}

\subsubsection{Haskell Scripts}
\begin{itemize}
  \item As well as the functions in the standard library, you can also define your own functions
  \item New functions are defined within a script, a text file comprising a sequence of definitions
  \item By convention, Haskell scripts usually have a .hs suffix on their filename. This is not mandatory, but is useful for identification purposes.
\end{itemize}

\subsubsection{My First Script}
\begin{lstlisting}[multicols=2]
double x = x + x
quadruple x = double (double x)
factorial n = product [1..n]
average ns = sum ns `div` length ns
\end{lstlisting}
Note:
\begin{itemize}
  \item \lstinline{div} is enclosed in back quotes, not forward
  \item \lstinline{x 'f' y} is just syntactic sugar for \lstinline{f x y}.
\end{itemize}

To start up GHCi with the script, type the following in a terminal:
\begin{lstlisting}[language=bash]
$ ghci test.hs
\end{lstlisting}
Now both the standard library and the file test.hs are loaded, and functions from both can be used:
\begin{lstlisting}[language=bash]
> quadruple 10
40
> take (double 2) [1,2,3,4,5,6]
[1,2,3,4]
\end{lstlisting}
GHCi does not automatically detect that the script has been changed, so a \lstinline[language=bash]{reload} command must be executed before the new definitions can be used:
\begin{lstlisting}[language=bash]
> :reload
Reading file "test.hs"
\end{lstlisting}

\subsubsection{Useful GHCi Commands}
\begin{tabularx}{\linewidth}{|X|X|}
  \hline
  \textbf{Command} & \textbf{Meaning} \\
  \hline
  \lstinline[language=bash]|:load name| & load script \lstinline[language=bash]|name| \\
  \lstinline[language=bash]|:reload| & reload current script \\
  \lstinline[language=bash]|:set editor name| & set editor to \lstinline[language=bash]|name| \\
  \lstinline[language=bash]|:edit name| & edit script \lstinline[language=bash]|name| \\
  \lstinline[language=bash]|:edit| & edit current script \\
  \lstinline[language=bash]|:type expr| & show type of \lstinline[language=bash]|expr| \\
  \lstinline[language=bash]|:?| & show all commands \\
  \lstinline[language=bash]|:quit| & quit GHCi \\
  \hline
\end{tabularx}

\subsubsection{Naming Requirements}
\begin{itemize}
  \item Function and argument names must begin with a lower-case letter:\\
  \lstinline{myFun}, \lstinline{fun1}, \lstinline{arg_2}, \lstinline{x'}
  \item By convention, list arguments usually have an s suffix on their name:\\
  \lstinline{xs}, \lstinline{ns}, \lstinline{nss}
\end{itemize}
%TODO: somewhere around here needs a newpage to be placed to solve problem with multicols
\subsubsection{The Layout Rule}
In a sequence of definitions, each definition must begin in precisely the same column:
\begin{multicols}{3}
correct:
\begin{lstlisting}
a = 10
b = 20
c = 30
\end{lstlisting}
\vfill\null
\columnbreak
wrong:
\begin{lstlisting}
a = 10
 b = 20
c = 30
\end{lstlisting}
\vfill\null
\columnbreak
wrong:
\begin{lstlisting}
 a = 10
b = 20
 c = 30
\end{lstlisting}
\vfill\null
\end{multicols}
The layout rule avoids the need for explicit syntax to indicate the grouping of definitions.
\begin{multicols}{2}
implicit grouping:
\begin{lstlisting}
a = b + c
    where
      b = 1
      c = 2
d = a * 2
\end{lstlisting}
\vfill\null
\columnbreak
explicit grouping:
\begin{lstlisting}
a = b + c
    where
    {b = 1;
     c = 2}
d = a * 2
\end{lstlisting}
\vfill\null
\end{multicols}

\subsection{Types and Classes}
\subsubsection{What is a Type?}
A type is a name for a collection of related values.
For example, in Haskell the basic type \lstinline{Bool} contains the two logical values \lstinline{False} and \lstinline{True}.

\subsubsection{Type Errors}
Applying a function to one or more arguments of the wrong type is called a type error.
\begin{multicols}{2}
  \begin{lstlisting}
  > 1 + False
  error ...
  \end{lstlisting}
  \lstinline{1} is a number and \lstinline{False} is a logical value, but \lstinline{+} requires two numbers.
\end{multicols}

\subsubsection{Types in Haskell}
\begin{itemize}
  \item If evaluating an expression \lstinline{e} would produce a value of type \lstinline{t}, then \lstinline{e} has type \lstinline{t}, written \lstinline{e :: t}
  \item Every well formed expression has a type, which can be automatically calculated at compile time using a process called type inference.
  \item All type errors are found at compile time, which makes programs safer and faster by removing the need for type checks at runtime.
  \item In GHCi, the \lstinline{:type} command calculates the type of an expression, without evaluating it:
\begin{lstlisting}
> not False
True
> :type not False
not False :: Bool
\end{lstlisting}
\end{itemize}

\subsubsection{Basic Types}
Haskell has a number of basic types, including:
\begin{tabularx}{\linewidth}{|X|X|}
  \hline
  \lstinline|Bool| & logical values \\
  \hline
  \lstinline|Char| & single characters \\
  \hline
  \lstinline|String| & strings of characters \\
  \hline
  \lstinline|Int| & integer numbers \\
  \hline
  \lstinline|Float| & floating-point numbers \\
  \hline
\end{tabularx}

\subsubsection{List Types}
\begin{lstlisting}
[False,True,False] :: [Bool]
['a','b','c','d'] :: [Char]
\end{lstlisting}
In general: \lstinline{[t]} is the type of lists with elements of type \lstinline{t}.

Note:
\begin{itemize}
  \item The type of a list says nothing about its length:
\begin{lstlisting}
  [False,True] :: [Bool]
  [False,True,False] :: [Bool]
\end{lstlisting}
  \item The type of the elements is unrestricted. For example, we can have lists of lists:
\begin{lstlisting}
[['a'],['b','c']] :: [[Char]]
\end{lstlisting}
\end{itemize}

\subsubsection{Tuple Types}
\begin{lstlisting}
(False,True) :: (Bool,Bool)
(False,'a',True) :: (Bool,Char,Bool)
\end{lstlisting}
In general: \lstinline{(t1,t2,...,tn)} is the type of n-tuples whose ith components have type ti for any i in 1..n.

Note:
\begin{itemize}
  \item The type of a tuple encodes its size:
\begin{lstlisting}
(False,True) :: (Bool,Bool)
(False,True,False) :: (Bool,Bool,Bool)
\end{lstlisting}
  \item The type of the components is unrestricted:
\begin{lstlisting}
('a',(False,'b')) :: (Char,(Bool,Char))
(True,['a','b']) :: (Bool,[Char])
\end{lstlisting}
\end{itemize}

\subsubsection{Function Types}
A function is a mapping from values of one type to values of another type:
\begin{lstlisting}
not :: Bool -> Bool
even :: Int -> Bool
\end{lstlisting}
In general: \lstinline{t1 -> t2} is the type of functions that map values of type \lstinline{t1} to values of type \lstinline{t2}.

Note:
\begin{itemize}
  \item The arrow \lstinline{->} is typed at the keyboard as \lstinline{->}.
  \item The argument and result types are unrestricted. For example, functions with multiple arguments or results are possible using list or tuples:
\begin{lstlisting}
add :: (Int,Int) -> Int
add (x,y) = x+y
zeroto :: Int -> [Int]
zeroto n = [0..n]
\end{lstlisting}
\end{itemize}

\subsubsection{Curried Functions}
Functions with multiple arguments are also possible by returning functions as results:
\begin{lstlisting}
add' :: Int -> (Int -> Int)
add' x y = x+y
\end{lstlisting}
\lstinline{add'} takes an integer \lstinline{x} and returns a function \lstinline{add' x}. In turn, this function takes an integer \lstinline{y} and returns the result \lstinline{x+y}.

Note:
\begin{itemize}
  \item \lstinline{add} and \lstinline{add'} produce the same final result, but \lstinline{add} takes its two arguments at the same time, whereas \lstinline{add'} takes them one at a time:
\begin{lstlisting}
add :: (Int,Int) -> Int
add' :: Int -> (Int -> Int)
\end{lstlisting}
  \item Functions that take their arguments one at a time are called curried functions, celebrating the work of Haskell Curry on such functions.
  \item Functions with more than two arguments can be curried by returning nested functions:
\begin{lstlisting}
mult :: Int -> (Int -> (Int -> Int))
mult x y z = x*y*z
\end{lstlisting}
\end{itemize}
\lstinline{mult} takes an integer \lstinline{x} and returns a function \lstinline{mult x}, which in turn takes an integer \lstinline{y} and returns a function \lstinline{mult x y}, which finally takes an integer \lstinline{z} and returns the result \lstinline{x*y*z}.

\subsubsection{Why is Currying Useful?}
Curried functions are more flexible than functions on tuples, because useful functions can often be made by partially applying a curried function.\\
For example:
\begin{lstlisting}
add' 1 :: Int -> Int
take 5 :: [Int] -> [Int]
drop 5 :: [Int] -> [Int]
\end{lstlisting}

\subsubsection{Currying Conventions}
To avoid excess parantheses when using curried functions, two simple conventions are adopted:
\begin{itemize}
  \item The arrow \lstinline{->} associates to the right.\\
        \lstinline{Int -> Int -> Int -> Int}\\
        Means \lstinline{Int -> (Int -> (Int -> Int))}.
\item As a consequence, it is then natural for function application to associate to the left.
      \lstinline{mult x y z}\\
      Means \lstinline{((mult x) y) z}.\\
      Unless tupling is explicitly required, all functions in Haskell are normally defined in curried form.
\end{itemize}

\subsubsection{Polymorphic Functions}
A function is called polymorphic (''of many forms'') if its type contains one or more type variables.
\begin{lstlisting}
length :: [a] -> Int
\end{lstlisting}
For any type \lstinline{a}, \lstinline{length} takes a list of values of type \lstinline{a} and returns an integer.

Note:
\begin{itemize}
  \item Type variables can be instantiated to different types in different circumstances:
\begin{lstlisting}
> length [False,True] -- a = Bool
2
> length [1,2,3,4] -- a = Int
4
\end{lstlisting}
  \item Many of the functions defined in the standard prelude are polymorphic. For example:
\begin{lstlisting}
fst :: (a,b) -> a
head :: [a] -> a
take :: Int -> [a] -> [a]
zip :: [a] -> [b] -> [(a,b)]
id :: a -> a
\end{lstlisting}
\end{itemize}

\subsubsection{Overloaded Functions}
A polymorpic function is called overloaded if its type contains one or more class constraints.
\begin{lstlisting}
(+) :: Num a => a -> a -> a
\end{lstlisting}
For any numeric type \lstinline{a}, \lstinline{(+)} takes two values of type \lstinline{a} and returns a value of type \lstinline{a}.

Note:
\begin{itemize}
  \item Constrained type variables can be instantiated to any types that satisfy the constraints:
\begin{lstlisting}
> 1 + 2 -- a = Int
3
> 1.0 + 2.0 -- a = Float
3.0
> 'a' + 'b' -- Char is not a numeric type
ERROR
\end{lstlisting}
  \item Haskell has a number of type classes, including:
\begin{tabularx}{\linewidth}{|X|X|}
  \hline
  \lstinline|Num| & Numeric types \\
  \hline
  \lstinline|Eq| & Equality types \\
  \hline
  \lstinline|Ord| & Ordered types \\
  \hline
\end{tabularx}
  \item For example:
\begin{lstlisting}
(+) :: Num a => a -> a -> a
(==) :: Eq a => a -> a -> Bool
(<) :: Ord a => a -> a -> Bool
\end{lstlisting}
\end{itemize}

\subsubsection{Hints and Tips}
\begin{itemize}
  \item When defining a new function in Haskell, it is useful to begin by writing down its type;
  \item Within a script, it is good practice to state the type of every new function defined;
  \item When stating the types of polymorphic functions that use numbers, equality or orderings, take care to include the necessary class constraints.
\end{itemize}

\subsection{Defining Functions}
\subsubsection{Conditional Expressions}
As in most programming languages, functions can be defined using conditional expressions.
\begin{lstlisting}
abs :: Int -> Int
abs n = if n >= 0 then n else -n
\end{lstlisting} %TODO: insert proper greater-than-or-equal sign
\lstinline{abs} takes an integer \lstinline{n} and returns \lstinline{n} if it is non-negative and \lstinline{-n} otherwise.

Conditional expressions can be nested:
\begin{lstlisting}
signum :: Int -> Int
signum n = if n < 0 then -1 else
              if n == 0 then 0 else 1
\end{lstlisting}

Note:
\begin{itemize}
  \item In Haskell, conditional expressions must always have an \lstinline{else} branch, which avoids any possible ambiguity problems with nested conditionals.
\end{itemize}

\subsubsection{Guarded Equations}
As an alternative to conditionals, functions can also be defined using guarded equations.
\begin{lstlisting}
abs n | n >= 0    = n
      | otherwise = -n
\end{lstlisting}
As previously, but using guarded equations.

Guarded equations can be used to make definitions involving multiple conditions easier to read:
\begin{lstlisting}
signum n | n < 0     = -1
         | n == 0    = 0
         | otherwise = 1
\end{lstlisting}
Note:
\begin{itemize}
  \item The catch all condition \lstinline{otherwise} is defined in prelude by \lstinline{otherwise = True}.
\end{itemize}

\subsubsection{Pattern Matching}
Many functions have a particularly clear definition using pattern matching on their arguments.
\begin{lstlisting}
not :: Bool -> Bool
not False = True
not True  = False
\end{lstlisting}
\lstinline{not} maps \lstinline{False} to \lstinline{True}, and \lstinline{True} to \lstinline{False}.

Functions can often be defined in many different ways using pattern matching. For example:
\begin{lstlisting}
(&&) :: Bool -> Bool -> Bool
True  && True  = True
True  && False = False
False && True  = False
False && False = False
\end{lstlisting}
can be defined more compactly by
\begin{lstlisting}
True && True = True
_    && _    = False
\end{lstlisting}
However, the following definition is more efficient, because it avoids evaluating the second argument if the first argument is \lstinline{False}:
\begin{lstlisting}
True  && b = b
False && _ = False
\end{lstlisting}

Note:
\begin{itemize}
  \item The underscore symbol \lstinline{_} is a wildcard pattern that matches any argument value.
  \item Patterns are matched in order. For example, the following definition always returns \lstinline{False}:
\begin{lstlisting}
_    && _    = False
True && True = True
\end{lstlisting}
  \item Patterns may not repeat variables. For example, the following definition gives an error:
\begin{lstlisting}
b && b = b
_ && _ = False
\end{lstlisting}
\end{itemize}

\subsubsection{List Patterns}
Internally, every non-empty list is constructed by repeated use of an operator (:) called ''cons'' that adds an element to the start of the list.
\begin{lstlisting}
[1,2,3,4]
\end{lstlisting}
Means \lstinline{1:(2:(3:(4:[])))}.

Functions on lists can be defined using \lstinline{x:xa} patterns.
\begin{lstlisting}
head :: [a] -> a
head (x:_) = x
tail :: [a] -> [a]
tail (_:xs) = xs
\end{lstlisting}
\lstinline{head} and \lstinline{tail} map any non-empty list to its first and remaining elements.

Note:
\begin{itemize}
  \item \lstinline{x:xs} patterns only match non-empty lists:
\begin{lstlisting}
> head []
*** Exception: empty list
\end{lstlisting}
  \item \lstinline{x:xs} patterns must be parenthesised, because application has priority over \lstinline{(:)}. For example, the following definition gives an error:
\begin{lstlisting}
head x:_ = x
\end{lstlisting}
\end{itemize}

\subsubsection{Lambda expressions}
Functions can be constructed without naming the functions by using lampda expressions.
\begin{lstlisting}
\x -> x + x
\end{lstlisting}
Note:
\begin{itemize}
  \item The symbol $\lambda$ is the Greek letter lambda, and is typed at the keyboard as a backslash \textbackslash.
  \item In mathematics, nameless functions are usually denoted using the $\mapsto$ symbol, as in $x \mapsto x + x$.
  \item In Haskell, the use of the $\lambda$ symbol for nameless functions comes from the lambda calculus, the theory of functions on which Haskell is based.
\end{itemize}

\subsubsection{Why are Lambda's useful?}
Lambda expressions can be used to give a formal meaning to functions defined using currying.

For example:
\begin{lstlisting}
add x y = x + y
\end{lstlisting}
means
\begin{lstlisting}
add = \x -> (\y -> x + y)
\end{lstlisting}

Lambda expressions can be used to avoid naming functions that are only referenced once.

For example:
\begin{lstlisting}
odds n = map f [0..n-1]
         where
            f x = x*2 + 1
\end{lstlisting}
can be simplified to
\begin{lstlisting}
odds n = map (\x -> x*2 + 1) [0..n-1]
\end{lstlisting}

\subsubsection{Operator Sections}
An operator written between its two arguments can be converted into a curried function written before its two arguments by using parentheses.

For example:
\begin{lstlisting}
> 1 + 2
3
> (+) 1 2
3
\end{lstlisting}

This convention also allows one of the arguments of the operator to be included in the parentheses.

For example:
\begin{lstlisting}
> (1+) 2
3
> (+2) 1
3
\end{lstlisting}

In general, if $\oplus$ is an operator then functions of the form $(\oplus)$, $(x\oplus)$ and $(\oplus y)$ are called sections.

\subsubsection{Why are Sections useful?}
Useful functions can sometimes be constructed in a simple way using sections.

For example:
\begin{itemize}
  \item[\-] \lstinline{(1+)} - successor function
  \item[\-] \lstinline{(1/)} - reciprocation function
  \item[\-] \lstinline{(*2)} - doubling function
  \item[\-] \lstinline{(/2)} - halving function    
\end{itemize}

\subsection{List Comprehensions}
\subsubsection{Set Comprehensions}
In mathematics, the comprehension notation can be used to construct new sets from old sets.

$\{x^2 | x \in \{1...5\}\}$
The set $\{1,4,9,16,25\}$ of all numbers $x^2$ such that $x$ is an element of the set $\{1...5\}$.

\subsubsection{Lists Comprehensions}
In Haskell, a similar comprehension notation can be used to construct new lists from old lists.
\begin{lstlisting}
[x^2 | x <- [1..5]]
\end{lstlisting}
The list \lstinline{[1,4,9,16,25]} of all numbers \lstinline{x^2} such that \lstinline{x} is an element of the list \lstinline{[1..5]}.

Note:
\begin{itemize}
  \item The expression \lstinline{x <- [1..5]} is called a generator, as it states how to generate values for \lstinline{x}.
  \item Comprehensions can have multiple generators, separated by commas. For example:
\begin{lstlisting}
> [(x,y) | x <- [1,2,3], y <- [4,5]]
[(1,4),(1,5),(2,4),(2,5),(3,4),(3,5)]
\end{lstlisting}
  \item Changing the order of the generators changes the order of the elements in the final list:
\begin{lstlisting}
> [(x,y) | y <- [4,5], x <- [1,2,3]]
[(1,4),(2,4),(3,4),(1,5),(2,5),(3,5)]
\end{lstlisting}
  \item Multiple generators are like nested loops, with later generators as more deeply nested loops whose variables change value more frequently.
  \item For example:
\begin{lstlisting}
> [(x,y) | y <- [4,5], x <- [1,2,3]]
[(1,4),(2,4),(3,4),(1,5),(2,5),(3,5)]
\end{lstlisting}
\lstinline{x <- [1,2,3]} is the last generator, so the value of the \lstinline{x} component of each pair changes most frequently.
\end{itemize}

\subsubsection{Dependant Generators}
Later generators can depend on the variables that are introduced by earlier generators.
\begin{lstlisting}
[(x,y) | x <- [1..3], y <- [x..3]]
\end{lstlisting}
The list \lstinline{[(1,1),(1,2),(1,3),(2,2),(2,3),(3,3)]} of all pairs of numbers \lstinline{(x,y,)} such that $x$,$y$ are elements of the list \lstinline{[1..3]} and $y \geq x$.

Using a dependant generator we can define the library function that concatenates a list of lists:
\begin{lstlisting}
concat :: [[a]] -> [a]
concat xss = [x | xs <- xss, x <- xs]
\end{lstlisting}

For example:
\begin{lstlisting}
> concat [[1,2,3],[4,5],[6]]
[1,2,3,4,5,6]
\end{lstlisting}

\subsubsection{Guards}
List comprehensions can use guards to restrict the values produced by earlier generators.
\begin{lstlisting}
[x | <- [1..10], even x]
\end{lstlisting}
The list \lstinline{[2,4,6,8,10]} of all numbers \lstinline{x} such that \lstinline{x} is an element of the list \lstinline{[1..10]} and \lstinline{x} is even.

Using a guard we can define a function that maps a positive integer to its list of factors:
\begin{lstlisting}
factors :: Int -> [Int]
factors n = 
    [x | y <- [1..n], n `mod` x == 0]
\end{lstlisting}

For example:
\begin{lstlisting}
> factors 15
[1,3,5,15]
\end{lstlisting}

A positive integer is prime if its only factors are 1 and itself.
Hence, using factors we can define a function that decides if a number is prime:
\begin{lstlisting}
prime :: Int -> Bool
prime n = factors n == [1,n]
\end{lstlisting}

For example:
\begin{lstlisting}
> prime 15
False
> prime 7
True
\end{lstlisting}

Using a guard we can now define a function that returns the list of all primes up to a given limit:
\begin{lstlisting}
primes :: Int -> [Int]
primes n = [x | x <- [2..n], prime x]
\end{lstlisting}

For example:
\begin{lstlisting}
> primes 40
[2,3,5,7,11,13,17,19,23,29,31,37]
\end{lstlisting}

\subsubsection{The Zip Function}
A useful library function is zip, which maps two lists to a list of pairs of their corresponding elements.
\begin{lstlisting}
zip :: [a] -> [b] -> [(a,b)]
\end{lstlisting}

For example:
\begin{lstlisting}
> zip ['a','b','c'] [1,2,3,4]
[('a',1),('b',2),('c',3)]
\end{lstlisting}

Using zip we can define a function returns that the list of all pairs of adjacent elements from a list:
\begin{lstlisting}
pairs :: [a] -> [(a,a)]
pairs xs = zip xs (tail xs)
\end{lstlisting}

For example:
\begin{lstlisting}
> pairs [1,2,3,4]
[(1,2),(2,3),(3,4)]
\end{lstlisting}

Using pairs we can define such a function that decides if the elements in a list are sorted:
\begin{lstlisting}
sorted :: Ord a => [a] -> Bool
sorted xs = and [x <= y | (x,y) <- pairs xs]
\end{lstlisting}

For example:
\begin{lstlisting}
> sorted [1,2,3,4]
True
> sorted [1,3,2,4]
False
\end{lstlisting}

Using zip we can define a function that returns the list of all positions of a value in a list:
\begin{lstlisting}
positions :: Eq a => a -> [a] -> [Int]
positions x xs = 
  [i | (x',i) <- zip xs [0..], x == x']
\end{lstlisting}

For example:
\begin{lstlisting}
> positions 0 [1,0,0,1,0,1,1,0]
[1,2,4,7]
\end{lstlisting}

\subsubsection{String Comprehensions}
A string is a sequence of characters enclosed in double quotes.
Internally, however, strings are represented as lists of characters.
\begin{lstlisting}
"abc" :: String
\end{lstlisting}
Means \lstinline{['a','b','c'] :: [Char]}.

Because strings are just special kinds of lists, any polymorphic function that operates on lists can also be applied to strings.
For example:
\begin{lstlisting}
> length "abcde"
5
> take 3 "abcde"
"abc"
> zip "abc" [1,2,3,4]
[('a',1),('b',2),('c',3)]
\end{lstlisting}

Similarly, list comprehensions can also be used to define functions on strings, such as counting how many times a character occurs in a string:
\begin{lstlisting}
count :: Char -> String -> Int
count x xs = length [x' | x' <- xs, x == x']
\end{lstlisting}

For example:
\begin{lstlisting}
> count 's' "Mississippi"
4
\end{lstlisting}

\subsection{Recursive Functions}
\subsubsection{Introduction}
As we have seen, many functions can naturally be defined in terms of other functions.
\begin{lstlisting}
fac :: Int -> Int
fac n = product [1..n]
\end{lstlisting}
\lstinline{fac} maps any integer \lstinline{n} to the product of the integers between \lstinline{1} and \lstinline{n}.

Expressions are evaluated by a stepwise process of applying functions to their arguments.

For example:
\begin{lstlisting}
fac 4
=
product [1..4]
=
product [1,2,3,4]
=
1*2*3*4
=
24
\end{lstlisting}

\subsubsection{Recursive Functions}
In Haskell, functions can also be defined in terms of themselves.
Such functions are called recursive.
\begin{lstlisting}
fac 0 = 1
fac n = n * fac (n-1)
\end{lstlisting}
\lstinline{fac} maps \lstinline{0} to \lstinline{1}, and any other integer to the product of itself and the factorial of its predecessor.

For example:
\begin{lstlisting}
fac 3
=
3 * fac 2
=
3 * (2 * fac 1)
=
3 * (2 * (1 * fac 0))
=
3 * (2 * (1 * 1))
=
3 * (2 * 1)
=
3 * 2
=
6
\end{lstlisting}

Note:
\begin{itemize}
  \item \lstinline{fac 0 = 1} is appropriate because \lstinline{1} is the identity for multiplication: \lstinline{1*x = x = x*1}.
  \item The recursive definition diverges on integers $< 0$ because the base case is never reached:
\begin{lstlisting}
> fac (-1)
*** Exception: stack overflow
\end{lstlisting}
\end{itemize}

\subsubsection{Why is Recursion useful?}
\begin{itemize}
  \item Some functions, such as \lstinline{factorial}, are simpler to define in terms of other functions.
  \item As we shall see, however, many functions can naturally be defined in terms of themselves.
  \item Properties of functions defined using recursion can be proved using the simple but powerful mathematical technique of induction.
\end{itemize}

\subsubsection{Recursion on Lists}
Recursion is not restricted to numbers, but can also be used to define functions on lists.
\begin{lstlisting}
product :: Num a => [a] -> a
product []     = 1
product (n:ns) = n * product ns
\end{lstlisting}
\lstinline{product} maps the empty list to \lstinline{1}, and any non-empty list to its head multiplied by the product of its tail.

For example:
\begin{lstlisting}
product [2,3,4]
=
2 * product [3,4]
=
2 * (3 * product [4])
=
2 * (3 * (4 * product []))
=
2 * (3 * (4 * 1))
=
24
\end{lstlisting}

Using the same pattern of recursion as in product we can define the \lstinline{length} function on lists.
\begin{lstlisting}
length :: [a] -> Int
length []     = 0
length (_:xs) = 1 + length xs
\end{lstlisting}
\lstinline{length} maps the empty list to \lstinline{0}, and any non-empty list to the successor of the length of its tail.

For example:
\begin{lstlisting}
length [1,2,3]
=
1 + length [2,3]
=
1 + (1 + length [3])
=
1 + (1 + (1 + length []))
=
1 + (1 + (1 + 0))
= 3
\end{lstlisting}

Using a similar pattern of recursion we can define the \lstinline{reverse} function on lists.
\begin{lstlisting}
reverse :: [a] -> [a]
reverse []     = []
reverse (x:xs) = reverse xs ++ [x]
\end{lstlisting}
\lstinline{reverse} maps the empty list to the empty list, and any non-empty list to the reverse of its tail appended to its head.

For example:
\begin{lstlisting}
reverse [1,2,3]
=
reverse [2,3] ++ [1]
=
(reverse [3] ++ [2]) ++ [1]
=
((reverse [] ++ [3]) ++ [2]) ++ [1]
=
(([] ++ [3]) ++ [2]) ++ [1]
=
[3,2,1]
\end{lstlisting}

\subsubsection{Multiple Arguments}
Functions with more than one argument can also be defined using recursion.
For example:
\begin{itemize}
  \item Zipping the elements of two lists:
\begin{lstlisting}
zip :: [a] -> [b] -> [(a,b)]
zip []     _      = []
Zip _      []     = []
zip (x:xs) (y:ys) = (x,y) : zip xs ys
\end{lstlisting}
  \item Remove the first \lstinline{n} elements from a list:
\begin{lstlisting}
drop :: Int -> [a] -> [a]
drop 0 xs     = xs
drop _ []     = []
drop n (_:xs) = drop (n-1) xs
\end{lstlisting}
  \item Appending two lists:
\begin{lstlisting}
(++) :: [a] -> [a] -> [a]
[]     ++ ys = ys
(x:xs) ++ ys = x : (xs ++ ys)
\end{lstlisting}
\end{itemize}

\subsubsection{Quicksort}
The quicksort algorithm for sorting a list of values can be specified by the following two rules:
\begin{itemize}
  \item The empty list is already sorted;
  \item Non-empty lists can be sorted by sorting the tail values $\leq$ the head, sorting the tail values $>$ the head, and then appending the resulting lists on either side of the head value.
\end{itemize}

Using recursion, this specification can be translated directly into an implementation:
\begin{lstlisting}
qsort :: Ord a => [a] -> [a]
qsort []     = []
qsort (x:xs) =
  qsort smaller ++ [x] ++ qsort larger
  where
    smaller = [a | a <- xs, a <= x]
    larger  = [b | b <- xs, b > x]
\end{lstlisting}

For example (abbreviating qsort as q):
%TODO: insert image

\subsection{Higher-Order Functions}
\subsubsection{Introduction}
A function is called higher-order if it takes a function as an argument or returns a function as a result.
\begin{lstlisting}
twice :: (a -> a) -> a -> a
twice f x = f (f x)
\end{lstlisting}
\lstinline{twice} is higher-order because it takes a function as its first argument.

\subsubsection{Why are they useful?}
\begin{itemize}
  \item Common programming idioms can be encoded as functions within the language itself.
  \item Domain specific languages can be defined as collections of higher-order functions.
  \item Algebraic properties of higher-order functions can be used to reason about programs.
\end{itemize}

\subsubsection{The \lstinline{map} function}
The higher-order library function called \lstinline{map} applies a function to every element of a list.
\begin{lstlisting}
map :: (a -> b) -> [a] -> [b]
\end{lstlisting}
For example:
\begin{lstlisting}
> map (+1) [1,3,5,7]
[2,4,6,8]
\end{lstlisting}

The \lstinline{map} function can be defined in a particularly simple manner using a list comprehension:
\begin{lstlisting}
map f xs = [f x | x <- xs]
\end{lstlisting}

Alternatively, for the purposes of proofs, the \lstinline{map} function can also be defined using recursion:
\begin{lstlisting}
map f []     = []
map f (x:xs) = f x : map f xs
\end{lstlisting} 

\subsubsection{The \lstinline{filter} function}
The higher-order library function \lstinline{filter} selects every element from a list that satisfies a predicate.
\begin{lstlisting}
filter :: (a -> Bool) -> [a] -> [a]
\end{lstlisting}
For example:
\begin{lstlisting}
> filter even [1..10]
[2,4,6,8,10]
\end{lstlisting}

Filter can be defined using a list comprehension:
\begin{lstlisting}
filter p xs = [x | x <- xs, p x]
\end{lstlisting}

Alternatively, it can be defined using recursion:
\begin{lstlisting}
filter p [] = []
filter p (x:xs)
  | p x       = x : filter p xs
  | otherwise = filter p x
\end{lstlisting}

\subsubsection{The \lstinline{foldr} function}
A number of functions on lists can be defined using the following simple pattern of recursion:
\begin{lstlisting}
f []     = v
f (x:xs) = x %*$\oplus$*) f xs
\end{lstlisting}
\lstinline{f} maps the empty list to some value \lstinline{v}, and any non-empty list to some function \(\oplus\) applied to its head and \lstinline{f} of its tail.

For example:\\
(\(v = 0; \oplus = +\))
\begin{lstlisting}
sum []     = 0
sum (x:xs) = x + sum xs
\end{lstlisting}

(\(v = 1; \oplus = *\))
\begin{lstlisting}
product []     = 1
product (x:xs) = x * product xs
\end{lstlisting}

(\(v = True; \oplus = \&\&\))
\begin{lstlisting}
and []     = True
and (x:xs) = x && and xs
\end{lstlisting}

The higher-order library function \lstinline{foldr} (fold right) encapsulates this simple pattern of recursion, with the function \(\oplus\) and the value \lstinline{v} as arguments.

For example:
\begin{lstlisting}
sum = foldr (+) 0
product = foldr (*) 1
or = foldr (||) False
and = foldr (&&) True
\end{lstlisting}

\lstinline{foldr} itself can be defined using recursion:
\begin{lstlisting}
foldr :: (a -> b -> b) -> b -> [a] -> b
foldr f v []     = v
foldr f v (x:xs) = f x (foldr f v xs)
\end{lstlisting}

However, it is best to think of \lstinline{foldr} non-recursively, as simultaneously replacing each \lstinline{(:)} in a list by a given function, and [] by a given value.

For example:\\
\begin{tabularx}{\linewidth}{lX}
  & \lstinline{sum [1,2,3]}\\
  \(=\) & \\
  & \lstinline{foldr (+) 0 [1,2,3]}\\
  \(=\) & \\
  & \lstinline{foldr (+) 0 (1:(2:(3:[])))}\\
  \(=\) & \tiny{(Replace each \lstinline{(:)} by \lstinline{(+)} and \lstinline{[]} by 0)} \\
  & \lstinline{1+(2+(3+0))}\\
  \(=\) & \\
  & \lstinline{6}\\
\end{tabularx}

For example:\\
\begin{tabularx}{\linewidth}{lX}
  & \lstinline{product [1,2,3]}\\
  \(=\) & \\
  & \lstinline{foldr (*) 1 [1,2,3]}\\
  \(=\) & \\
  & \lstinline{foldr (*) 1 (1:(2:(3:[])))}\\
  \(=\) & \tiny{(Replace each \lstinline{(:)} by \lstinline{(*)} and \lstinline{[]} by 1)} \\
  & \lstinline{1*(2*(3*1))}\\
  \(=\) & \\
  & \lstinline{6}\\
\end{tabularx}

\subsubsection{Other \lstinline{foldr} examples}
Even though \lstinline{foldr} encapsulates a simple pattern of recursion, it can be used to define many more functions than might first be expected.

Recall the length function:
\begin{lstlisting}
length :: [a] -> Int
length []     = 0
length (_:xs) = 1 + length xs
\end{lstlisting}

For example:\\
\begin{tabularx}{\linewidth}{lX}
  & \lstinline{length [1,2,3]}\\
  \(=\) & \\
  & \lstinline{length (1:(2:(3:[])))}\\
  \(=\) & \tiny{(Replace each \lstinline{(:)} by \(\lambda_n \rightarrow 1 + n\) and \lstinline{[]} by 0)} \\
  & \lstinline{1+(1+(1+0))}\\
  \(=\) & \\
  & \lstinline{3}\\
\end{tabularx}

Hence, we have:\\
\lstinline{length = foldr (}\(\lambda_n \rightarrow 1+n\)\lstinline{) 0}

Now recall the \lstinline{reverse} function:
\begin{lstlisting}
reverse []     = []
reverse (x:xs) = reverse xs ++ [x]
\end{lstlisting}

For example:\\
\begin{tabularx}{\linewidth}{lX}
  & \lstinline{reverse [1,2,3]}\\
  \(=\) & \\
  & \lstinline{reverse (1:(2:(3:[])))}\\
  \(=\) & \tiny{(Replace each \lstinline{(:)} by \(\lambda\)\lstinline{x xs}\(\rightarrow\)\lstinline{xs ++ [x]} and \lstinline{[]} by \lstinline{[]})}\\
  & \lstinline{(([] ++ [3]) ++ [2]) ++ [1]}\\
  \(=\) & \\
  & \lstinline{[3,2,1]}\\
\end{tabularx}

Hence, we have:\\
\lstinline{reverse = foldr (}\(\lambda\)\lstinline{x xs }\(\rightarrow\)\lstinline{ xs ++ [x]) []}

Finally, we note that the append function \lstinline{(++)} has a particularly compact definition using \lstinline{foldr}:\\
\lstinline{(++ ys) = foldr (:) ys} {\tiny{(Replace each \lstinline{(:)} by \lstinline{(:)} and \lstinline{[]} by \lstinline{ys}.)}}

\subsubsection{Why is \lstinline{foldr} useful?}
\begin{itemize}
  \item Some recursive functions on lists, such as \lstinline{sum}, are simpler to define using \lstinline{foldr}.
  \item Properties of functions defined using \lstinline{foldr} can be proved using algebraic properties of \lstinline{foldr}, such as fusion and the banana split rule.
  \item Advanced program optimizations can be simpler if \lstinline{foldr} is used in place of explicit recursion.
\end{itemize}

\subsubsection{Other library functions}
The library function \lstinline{(.)} returns the composition of two functions as a single function.
\begin{lstlisting}
(.) :: (b -> c) -> (a -> b) -> (a -> c)
f . g = \x -> f (g x)
\end{lstlisting}

For example:
\begin{lstlisting}
odd :: Int -> Bool
odd = not . even
\end{lstlisting}

The library function \lstinline{all} decides if every element of a list satisfies a given predicate.
\begin{lstlisting}
all :: (a -> Bool) -> [a] -> Bool
all p xs = and [p x | x <- xs]
\end{lstlisting}

For example:
\begin{lstlisting}
> all even [2,4,6,8,10]
True
\end{lstlisting}

Dually, the library function \lstinline{any} decides if at least one element of a list satisfies a predicate.
\begin{lstlisting}
any :: (a -> Bool) -> [a] -> Bool
any p xs = or [p x | x <- xs]
\end{lstlisting}

For example:
\begin{lstlisting}
> any (== ' ') "abc def"
True
\end{lstlisting}

The library function \lstinline{takeWhile} selects elements from a list while a predicate holds of all the elements.
\begin{lstlisting}
takeWhile :: (a -> Bool) -> [a] -> [a]
takeWhile p [] = []
takeWhile p (x:xs)
  | p x       = x : takeWhile p xs
  | otherwise = []
\end{lstlisting}

For example:
\begin{lstlisting}
> takeWhile (/= ' ') "abc def"
"abc"
\end{lstlisting}

Dually, the function \lstinline{dropWhile} removes elements while a predicate holds of all the elements.
\begin{lstlisting}
dropWhile :: (a -> Bool) -> [a] -> [a]
dropWhile p [] = []
dropWhile p (x:xs)
  | p x       = dropWhile p xs
  | otherwise = x:xs
\end{lstlisting}

For example:
\begin{lstlisting}
> dropWhile (== ' ') "    abc"
"abc"
\end{lstlisting}


\subsection{Declaring Types and Classes}
\subsubsection{Type declarations}
In Haskell, a new name for an existing type can be defined using a type declaration.
\begin{lstlisting}
type String = [Char]
\end{lstlisting}
String is a synonym for the type \lstinline{[Char]}.

Type declarations can be used to make other types easier to read.
For example, given
\begin{lstlisting}
type Pos = (Int, Int)
\end{lstlisting}
we can define:
\begin{lstlisting}
origin :: Pos
origin = (0,0)
left :: Pos -> Pos
left (x,y) = (x-1,y)
\end{lstlisting}

Like function definitions, type declarations can also have parameters.
For example, given
\begin{lstlisting}
type Pair a = (a,a)
\end{lstlisting}
we can define:
\begin{lstlisting}
mult :: Pair Int -> Int
mult (m,n) = m*n
copy :: a -> Pair a
copy x = (x,x)
\end{lstlisting}

Type declarations can be nested:
\begin{lstlisting}[frame=single,rulecolor=\color{green}]
type Pos = (Int, Int)
type Trans = Pos -> Pos
\end{lstlisting}

However, they \textbf{cannot} be recursive:
\begin{lstlisting}[frame=single,rulecolor=\color{red}]
type Tree = (Int,[Tree])
\end{lstlisting}

\subsubsection{Data declarations}
A completely new type can be defined by specifying its values using a data declaration.
\begin{lstlisting}
data Bool = False | True
\end{lstlisting}
\lstinline{Bool} is a new type, with two new values \lstinline{False} and \lstinline{True}.

Note:
\begin{itemize}
  \item The two values \lstinline{False} and \lstinline{True} are called the constructors for the type \lstinline{Bool}.
  \item Type and constructor names must always begin with an upper-case letter.
  \item Data declarations are similar to context free grammars. The former specifies the values of a type, the latter the sentences of a language.
\end{itemize}

Values of new types can be used in the same ways as those of built in types.
For example, given
\begin{lstlisting}
data Answer = Yes | No | Unknown
\end{lstlisting}
we can define:
\begin{lstlisting}
answers :: [Answer]
answers = [Yes,No,Unknown]
flip :: Answer -> Answer
flip Yes     = No
flip No      = Yes
flip Unknown = Unknown
\end{lstlisting}

The constructors in a data declaration can also have parameters.
For example, given
\begin{lstlisting}
data Shape = Circle Float
  | Rect Float Float
\end{lstlisting}
we can define:
\begin{lstlisting}
square :: Float -> Shape
square n = Rect n n
area :: Shape -> Float
area (Circle r) = pi * r^2
area (Rect x y) = x * y
\end{lstlisting}

Note:
\begin{itemize}
  \item \lstinline{Shape} has values of the form \lstinline{Circle r} where \lstinline{r} is a float, and \lstinline{Rect x y} where \lstinline{x} and \lstinline{y} are floats.
  \item \lstinline{Circle} and \lstinline{Rect} can be viewed as functions that construct values of type \lstinline{Shape}:
  \begin{lstlisting}
  Circle :: Float -> Shape
  Rect :: Float -> Float -> Shape
  \end{lstlisting}
\end{itemize}

Not surprisingly, data declarations themselves can also have parameters.
For example, given
\begin{lstlisting}
data Maybe a = Nothing | Just a
\end{lstlisting}
we can define:
\begin{lstlisting}
safediv == Int -> Int -> Maybe Int
safediv _ 0 = Nothing
safediv m n = Just (m `div` n)

safehead :: [a] -> Maybe a
safehead [] = Nothing
safehead xs = Just (head xs)
\end{lstlisting}

\subsubsection{Recursive types}
In Haskell, new types can be declared in terms of themselves.
That is, types can be recursive.
\begin{lstlisting}
data Nat = Zero | Succ Nat
\end{lstlisting}
\lstinline{Nat} is a new type, with constructors \lstinline{Zero :: Nat} and \lstinline{Succ :: Nat -> Nat}.

Note:
\begin{itemize}
  \item A value of type \lstinline{Nat} is either \lstinline{Zero}, or of the form \lstinline{Succ n} where \lstinline{n :: Nat}. That is, \lstinline{Nat} contains the following infinite sequences of values:
\begin{lstlisting}
Zero
Succ Zero
Succ (Succ Zero)
%*\(\vdots\)*)
\end{lstlisting}
  \item We can think of values of type \lstinline{Nat} as natural numbers, where \lstinline{Zero} represents \lstinline{0}, and \lstinline{Succ} represents the successor function \lstinline{1+}.
  \item For example, the value
\begin{lstlisting}
Succ (Succ (Succ Zero))
\end{lstlisting}
  represents the natural number
\begin{lstlisting}
1 + (1 + (1 + 0)) = 3
\end{lstlisting}
\end{itemize}

Using recursion, it is easy to define functions that convert between values of type \lstinline{Nat} and \lstinline{Int}:
\begin{lstlisting}
nat2int :: Nat -> Int
nat2int Zero     = 0
nat2int (Succ n) = 1 + nat2int n

int2nat :: Int -> Nat
int2nat 0 = Zero
int2nat n = Succ (int2nat (n-1))
\end{lstlisting}

Two naturals can be added by converting them to integers, adding, and then converting back:
\begin{lstlisting}
add :: Nat -> Nat -> Nat
add m n = int2nat (nat2int m + nat2int n)
\end{lstlisting}

However, using recursion the function \lstinline{add} can be defined without the need for conversions:
\begin{lstlisting}
add Zero     n = n
add (Succ m) n = Succ (add m n)
\end{lstlisting}

For example:
\begin{tabularx}{\linewidth}{lX}
  & \lstinline{add (Succ (Succ Zero)) (Succ Zero)}\\
  \(=\) & \\
  & \lstinline{Succ (add (Succ Zero)) (Succ Zero)}\\
  \(=\) & \\
  & \lstinline{Succ (Succ (add Zero (Succ Zero)))}\\
  \(=\) & \\
  & \lstinline{Succ (Succ (Succ Zero))}\\
\end{tabularx}

Note:
\begin{itemize}
  \item The recursive definition for \lstinline{add} corresponds to the laws \lstinline{0+n=n} and \lstinline{(1+m)+n = 1+(m+n)}.
\end{itemize}

\subsubsection{Arithmetic expressions}
Consider a simple form of expressions built up from integers using addition and multiplication.
%TODO: insert image

Using recursion, a suitable new type to represent such expressions can be declared by:
\begin{lstlisting}
data Expr = Val Int
          | Add Expr Expr
          | Mul Expr Expr
\end{lstlisting}

For example, the expression on the previous slide would be represented as follows:
\begin{lstlisting}
Add (Val 1) (Mul (Val 2) (Val 3))
\end{lstlisting}

Using recursion, it is now easy to define functions that process expressions.
For example:
\begin{lstlisting}
size :: Expr -> Int
size (Val n)   = 1
size (Add x y) = size x + size y
size (Mul x y) = size x + size y

eval :: Expr -> Int
eval (Val n)   = n
eval (Add x y) = eval x + eval y
eval (Mul x y) = eval x * eval y
\end{lstlisting}

Note:
\begin{itemize}
  \item The three constructors have types:
\begin{lstlisting}
Val :: Int -> Expr
Add :: Expr -> Expr -> Expr
Mul :: Expr -> Expr -> Expr
\end{lstlisting}
  \item Many functions on expressions can be defined by replacing the constructors by other functions using a suitable \lstinline{fold} function. For example:
\begin{lstlisting}
eval folde id (+) (*)
\end{lstlisting}
\end{itemize}


\subsection{The Countdown Problem}
\subsubsection{What is Countdown?}
\begin{itemize}
  \item A popular quiz programme on British television that has been running since 1982.
  \item Includes a numbers game that we shall refer to as the countdown problem.
\end{itemize}

\subsubsection{Example}
Using the numbers
\begin{lstlisting}
1 3 7 10 25 50
\end{lstlisting}
and the arithmetic operators
\begin{lstlisting}
+ - * /
\end{lstlisting}
construct an expression whose value is \lstinline{765}.

\subsubsection{Rules}
\begin{itemize}
  \item All the numbers, including intermediate results, must be positive naturals \((1,2,3,\ldots)\).
  \item Each of the source numbers can be used at most once when constructing the expression.
  \item We abstract from other rules that are adopted on television for pragmatic reasons.
\end{itemize}

For our example, one possible solution is
\begin{lstlisting}
(25 - 10) * (50+1) = 765
\end{lstlisting}

Notes:
\begin{itemize}
  \item There are \(780\) solutions for this example.
  \item Changing the target number to \lstinline{831} gives an example that has no solutions.
\end{itemize}

\subsubsection{Evaluating Expressions}
Operators:
\begin{lstlisting}
data Op = Add | Sub | Mul | Div
\end{lstlisting}
Apply an operator:
\begin{lstlisting}
apply :: Op -> Int -> Int -> Int
apply Add x y = x + y
apply Sub x y = x - y
apply Mul x y = x * y
apply Div x y = x `div` y
\end{lstlisting}

Decide if the result of applying an operator to two positive natural numbers is another such:
\begin{lstlisting}
valid :: Op -> Int -> Int -> Bool
valid Add _ _ = True
valid Sub x y = x > y
valid Mul _ _ = True
valid Div x y = x `mod` y == 0
\end{lstlisting}

Expression:
\begin{lstlisting}
data Expr = Val Int | App Op Expr Expr
\end{lstlisting}

Return the overall value of an expression, provided that it is a positive natural number:
\begin{lstlisting}
eval :: Expr -> [Int]
eval (Val n)
eval ( App o l r) = [apply o x y | x <- eval l
                                 , y <- eval r
                                 , valid o x y]
\end{lstlisting}
Either succeeds and returns a singleton list, or fails and returns the empty list.

\subsubsection{Formalizing the problem}
Return a list of all possible ways of choosing zero or more elements from a list:
\begin{lstlisting}
choices :: [a] -> [[a]]
\end{lstlisting}

For example:
\begin{lstlisting}
> choices [1,2]
[[],[1],[2],[1,2],[2,1]]
\end{lstlisting}

Return a list of all the values in an expression:
\begin{lstlisting}
values :: Expr -> [Int]
values (Val n)     = [n]
values (App _ l r) = values l ++ values r
\end{lstlisting}

Decide if an expression is a solution for a given list of source numbers and a target number:
\begin{lstlisting}
solution :: Expr -> [Int] -> Int -> Bool
solution e ns n = elem (values e) (choices ns)
                  && eval e == [n]
\end{lstlisting}

\subsubsection{Brute force solution}
Return a list of all possible ways of splitting a list into two non-empty parts:
\begin{lstlisting}
split :: [a] -> [([a],[a])]
\end{lstlisting}

For example:
\begin{lstlisting}
> split [1,2,3,4]
[([1], [2,3,4]),([1,2],[3,4]),([1,2,3],[4])]
\end{lstlisting}

Return a list of all possible expressions whose values are precisely a given list of numbers:
\begin{lstlisting}
exprs :: [Int] -> [Expr]
exprs []  = []
exprs [n] = [Val n]
exprs ns  = [e | (ls,rs) <- split ns
               , l       <- exprs ls
               , r       <- exprs rs
               , e       <- combine l r]
\end{lstlisting}
The key function in this lecture.

Combine two expressions using each operator:
\begin{lstlisting}
combine :: Expr -> Expr -> [Epxr]
combine l r =
  [App o l r | o <- [Add,Sub,Mul,Div]]
\end{lstlisting}

Return a list of all possible expressions that solve an instance of the countdown problem:
\begin{lstlisting}
solutions :: [Int] -> Int -> [Expr]
solutions ns n = [e | ns' <- choices ns
                    , e   <- exprs ns'
                    , eval e == [n]]
\end{lstlisting}

\subsubsection{How fast is it?}
\begin{description}
  \item[System] 2.8GHz Core 2 Duo, 4GB RAM
  \item[Compiler] GHC version 7.10.2
  \item[Example] \lstinline{solutions [1,3,7,10,25,50] 765}
  \item[One solution] 0.108 seconds
  \item[All solutions] 12.224 seconds     
\end{description}

\subsubsection{Can we do better?}
\begin{itemize}
  \item Many of the expressions that are considered will typically be invalid - fail to evaluate.
  \item For our example, only around 5 million of the 33 million possible expressions are valid.
  \item Combining generation with evaluation would allow earlier rejection of invalid expressions.
\end{itemize}

\subsubsection{Fusing two functions}
Valid expressions and their values:
\begin{lstlisting}
type Result = (Expr,Int)
\end{lstlisting}

We seek to define a function that fuses together the generation and evaluation of expressions:
\begin{lstlisting}
results :: [Int] -> [Result]
results ns = [(e,n) | e <- exprs ns
                    , n <- eval e]
\end{lstlisting}

This behaviour is achieved by defining
\begin{lstlisting}
results []  = []
results [n] = [(Val n,n) | n > 0]
results ns =
  [res | (ls,rs) <- split ns
       , lx      <- results ls
       , ry      <- results rs
       , res     <- combine' lx ry]
\end{lstlisting}
where
\begin{lstlisting}
combine' :: Result -> Result -> [Result]
\end{lstlisting}

Combining results:
\begin{lstlisting}
combine' (l,x) (r,y) =
  [(App o l r, apply o x y)
    | o <- [Add,Sub,Mul,Div]
    , valid o x y]
\end{lstlisting}
New function that solves countdown problems:
\begin{lstlisting}
solutions' :: [Int] -> Int -> [Expr]
solutions' ns n =
  [e | ns'   <- choices ns
     , (e,m) <- results ns'
     , m == n]
\end{lstlisting}

\subsubsection{How fast is it now?}
\begin{description}
  \item[Example] \lstinline{solutions' [1,3,7,10,25,50] 765}
  \item[One solution] 0.014 seconds
  \item[All solutions] 1.312 seconds   
\end{description}
Around 10 times faster in both cases.

\subsubsection{Can we do better?}
\begin{itemize}
  \item Many expressions will be essentially the same using simple arithmetic properties, such as:
\begin{lstlisting}
x * y = y * x
x * 1 = x
\end{lstlisting}
  \item Exploiting such properties would considerably reduce the search and solution spaces.
\end{itemize}

\subsubsection{Exploiting properties}
Strengthening the valid predicate to take account of commutativity and identity properties:
\begin{lstlisting}
valid :: Op -> Int -> Int -> Bool
valid Add x y = x <= y
valid Sub x y = x > y
valid Mul x y = x <= y && x /= 1 && y /= 1
valid Div x y = x `mod` y == 0 && y /= 1
\end{lstlisting}

\subsubsection{How fast is it now?}
\begin{description}
  \item[Example] \lstinline{solutions'' [1,3,7,10,25,50] 765}
  \item[Valid] 250000 expressions (around 20 times less)
  \item[Solutions] 49 expressions (around 16 times less)
  \item[One solution] 0.007 seconds (around 2 times faster)
  \item[All solutions] 0.119 seconds (around 11 times faster)   
\end{description}

More generally, our program usefully returns all solutions in a fraction of a second, and is around 100 times faster than the original version.


\subsection{Lazy evaluation}
\subsubsection{Introduction}
Expressions in Haskell are evaluated using a simple technique called lazy evaluation, which:
\begin{itemize}
  \item Avoids doing unnecessary evaluation;
  \item Ensures termination whenever possible;
  \item Supports programming with infinite lists;
  \item Allows programs to be more modular.
\end{itemize}

\subsubsection{Evaluating expressions}
\begin{lstlisting}
square n = n * n
\end{lstlisting}

Example:
\begin{tabularx}{\linewidth}{lX}
  & \lstinline{square (1+2)}\\
  \(=\) & \tiny{Apply \lstinline{+} first.}\\
  & \lstinline{square 3}\\
  \(=\) & \\
  & \lstinline{3 * 3}\\
  \(=\) & \\
  & \lstinline{9}\\
\end{tabularx}

Another evaluation order is also possible:
\begin{tabularx}{\linewidth}{lX}
  & \lstinline{square (1+2)}\\
  \(=\) & \tiny{Apply \lstinline{square} first.}\\
  & \lstinline{(1+2) * (1+2)}\\
  \(=\) & \\
  & \lstinline{3 * (1+2)}\\
  \(=\) & \\
  & \lstinline{3 * 3}\\
  \(=\) & \\
  & \lstinline{9}\\
\end{tabularx}

Any way of evaluating the same expression will give the same result, provided it terminates.

\subsubsection{Evaluation strategies}
There are two main strategies for decoding which reducible expression (redex) to consider next:
\begin{itemize}
  \item Choose a redex that is innermost, in the sense that does not contain another redex;
  \item Choose a redex that is outermost, in the sense that it is not contained in another redex.
\end{itemize}

\subsubsection{Termination}
\begin{lstlisting}
infinity = 1 + infinity
\end{lstlisting}

Example:\\
\begin{tabularx}{\linewidth}{lX}
  & \lstinline{fst (0,infinity)}\\
  \(=\) & \tiny{Innermost evaluation.}\\
  & \lstinline{fst (0,1+infinity)}\\
  \(=\) & \\
  & \lstinline{fst (0,1+(1+infinity))}\\
  \(=\) & \\
  & \(\vdots\)\\
  \(=\) & \\
  & \lstinline{fst (0,infinity)}\\
  \(=\) & \tiny{Outermost evaluation.}\\
  & \lstinline{0}\\
\end{tabularx}

Note:
\begin{itemize}
  \item Outermost evaluation may give a result when innermost evaluation fails to terminate;
  \item If any evaluation sequence terminates, then so does outermost, with the same result.
\end{itemize}

\subsubsection{Number of reductions}
For \lstinline{square} example:
\begin{itemize}
  \item Innermost: 3 steps
  \item Outermost: 4 steps
\end{itemize}

Note:
\begin{itemize}
  \item The outmost version is inefficient, because the argument \lstinline{1+2} is duplicated when \lstinline{square} is applied and is hence evaluated twice.
  \item Due to such duplication, outermost evaluation may require more steps than innermost.
  \item This problem can easily be avoided by using pointers to indicate sharing of arguments.
\end{itemize}

Example:\\
\begin{tabularx}{\linewidth}{lX}
  & \lstinline{square (1+2)}\\
  \(=\) & \\
  & \lstinline{* 1+2}\\
  \(=\) & \\
  & \lstinline{* 3}\\
  \(=\) & \tiny{Shared argument evaluated once.}\\
  & \lstinline{9}\\
\end{tabularx}

This gives a new evaluation strategy:\\
lazy evaluation = outermost evaluation \(+\) sharing of arguments

Note:
\begin{itemize}
  \item Lazy evaluation ensures termination whenever possible, but never requires more steps than innermost evaluation and sometimes fewer.
\end{itemize}

\subsubsection{Infinite lists}
\begin{lstlisting}
ones = 1 : ones
\end{lstlisting}

Example:
\begin{tabularx}{\linewidth}{lX}
  & \lstinline{ones}\\
  \(=\) & \\
  & \lstinline{1 : ones}\\
  \(=\) & \\
  & \lstinline{1 : (1 : ones)}\\
  \(=\) & \\
  & \(\vdots\)\\
  \(=\) & \tiny{An infinite list of ones.}\\
\end{tabularx}

What happens if we select the first element?
\begin{itemize}
  \item Innermost: does not terminate
  \item Lazy: terminates in 2 steps
\end{itemize}

Note:
\begin{itemize}
  \item In the lazy case, only the first element of ones is produced, as the rest are not required.
  \item In general, with lazy evaluation expressions are only evaluated as much as required by the context in which they are used.
  \item Hence, \lstinline{ones} is really a potentially infinite list.
\end{itemize}

\subsubsection{Modular programming}
Lazy evaluation allows us to make programs more modular by separation control from data.
\begin{lstlisting}
> take 5 ones
[1,1,1,1,1]
\end{lstlisting}
The data part \lstinline{ones} is only evaluated as much as required by the control part \lstinline{take 5}.

Whitout using lazy evaluation the control and data parts would need to be combined into one:
\begin{lstlisting}
replicate :: Int -> a -> [a]
replicate 0 _ = []
replicate n x = x : replicate (n-1) x
\end{lstlisting}

Example:\\
\begin{lstlisting}
> replicate 5 1
[1,1,1,1,1]
\end{lstlisting}

\subsubsection{Generating primes}
To generate the infininte sequence of primes:
\begin{enumerate}
  \item Write down the infinite sequence \(2,3,4,\ldots\);
  \item Mark the first number \(p\) as being prime;
  \item Delete all multiples of \(p\) from the sequence;
  \item Return to seconde step.
\end{enumerate}

This idea can be driectly translated into a program that generates the infinite list of primes!
\begin{lstlisting}
primes :: [Int]
primes = sieve[2..]

sieve :: [Int] -> [Int]
sieve (p:xs) = 
  p : sieve [x | x <- xs, mod x p /= 0]
\end{lstlisting}

Examples:
\begin{lstlisting}
> primes
[2,3,5,7,11,13,17,19,..]
\end{lstlisting}

\begin{lstlisting}
> take 10 primes
[2,3,5,7,11,13,17,19,23,29]
\end{lstlisting}

\begin{lstlisting}
> takeWhile (<10) primes
[2,3,5,7]
\end{lstlisting}

We can also use \lstinline{primes} to generate an (infinite?) list of twin primes that differ by precisely two.
\begin{lstlisting}
twin :: (Int,Int) -> Int
twin (x,y) = y == x+2
\end{lstlisting}

\begin{lstlisting}
twins :: [(Int,Int)]
twins = filter twin (zip primes (tail primes))
\end{lstlisting}

\begin{lstlisting}
> twins
[(3,5),(5,7),(11,13),(17,19),(29,31),..]
\end{lstlisting}


\newpage
\subsection{Exercises:}

\subsubsection{Exercise 1}
\textbf{Types of Lists and Tuples}:\\
Given the declaration \lstinline|x = 'x'|, which expressions are correctly typed?\\
\begin{tabularx}{\linewidth}{|X|X|}
  \hline
  \lstinline|x| & \lstinline|Char|\\
  \hline
  \lstinline|'x'| & \lstinline|Char|\\
  \hline
  \lstinline|"x"| & \lstinline|[Char]|\\
  \hline
  \lstinline|['x']| & \lstinline|[Char]|\\
  \hline
  \lstinline|[x, 'x']| & \lstinline|[Char]|\\
  \hline
  \lstinline|[x, x, x, x]| & \lstinline|[Char]|\\
  \hline
  \lstinline|['x', "x"]| & \textbf{Not correct.}\\
  \hline
  \lstinline|[x == 'x', True]| & \lstinline|[Bool]|\\
  \hline
  \lstinline|[["True"]]| & \lstinline|[[[Char]]]|\\
  \hline
  \lstinline|[[True, False], True]| & \textbf{Not correct.}\\
  \hline
  \lstinline|[[True, False], []]| & \lstinline|[[Bool]]|\\
  \hline
  \lstinline|('x')| & \lstinline|Char|\\
  \hline
  \lstinline|(x, 'x')| & \lstinline|(Char, Char)|\\
  \hline
  \lstinline|(x, x, x, x)| & \lstinline|(Char, Char, Char, Char)|\\
  \hline
  \lstinline|('x', "x")| & \lstinline|(Char, [Char])|\\
  \hline
  \lstinline|(x, True)| & \lstinline|(Char, Bool)|\\
  \hline
  \lstinline|(x == 'x', True)| & \lstinline|(True, True)|\\
  \hline
  \lstinline|(("True"))| & \lstinline|[Char]|\\
  \hline
  \lstinline|((True, False), True)| & \lstinline|((Bool, Bool), Bool)|\\
  \hline
  \lstinline|((True, False), ())| & \lstinline|((Bool, Bool), ())|\\
  \hline
\end{tabularx}

\textbf{Types of Lists}\\
Given the declaration \lstinline|a = [True]|, which expressions are correctly typed?\\
\begin{tabularx}{\linewidth}{|X|X|}
  \hline
  \lstinline|a| & \lstinline|[Bool]|\\
  \hline
  \lstinline|a ++ a ++ [True]| & \lstinline|[Bool]|\\
  \hline
  \lstinline|a ++ []| & \lstinline|[Bool]|\\
  \hline
  \lstinline|head a| & \lstinline|Bool|\\
  \hline
  \lstinline|tail a| & \lstinline|[Bool]|\\
  \hline
  \lstinline|head 'x'| & \textbf{Not correct.}\\
  \hline
  \lstinline|head "x"| & \lstinline|Char|\\
  \hline
  \lstinline|tail "x"| & \lstinline|[Char]|\\
  \hline
  \lstinline|"dimdi" !! 2| & \lstinline|Char|\\
  \hline
  \lstinline|"dimdi" ++ "ding"| & \lstinline|[Char]|\\
  \hline
\end{tabularx}

\textbf{Types of Lists and Tuples Mixed}\\
Given the declaration:
\begin{lstlisting}
einkaufsliste =
  [(3, "Widerstand 10kOhm"),
   (5, "Kondensator 0.1microFarad"),
   (2, "Zahnrad 38 Zaehne")]
preisliste =
  [("Zahnrad 38 Zaehne", 1200),
   ("Widerstand 10 kOhm", 50),
   ("Widerstand 20 kOhm", 50),
   ("Kondensator 0.1microFarad", 50)]
\end{lstlisting}
Which expressions are correctly typed?\\
\begin{tabularx}{\linewidth}{|X|X|}
  \hline
  \lstinline|[(True, 'a'), (False, 'b')]| & \lstinline|[(Bool, Char)]|\\
  \hline
  \lstinline|[(True, 'a'), ('b', False)]| & \textbf{Not correct.}\\
  \hline
  \lstinline|[(True, 'a'), ('a' == 'b', head "a")]| & \lstinline|[(Bool, Char)]|\\
  \hline
  \lstinline|([True, 'a' == 'b'], ['a'])| & \lstinline|([Bool],[Char])|\\
  \hline
  \lstinline|('a', 'b', 'c', 'd')| & \lstinline|(Char, Char, Char, Char)|\\
  \hline
  \lstinline|('a', ('b', ('c', ('d'))))| & \lstinline|(Char, (Char, (Char, Char)))|\\
  \hline
  \lstinline|(('a', 'b'), 'c', 'd')| & \lstinline|((Char, Char), Char, Char)|\\
  \hline
  \lstinline|['a', 'b', 'c', 'd']| & \lstinline|[Char]|\\
  \hline
  \lstinline|einkaufsliste| & \lstinline|[(Integer, [Char])]|\\
  \hline
  \lstinline|preisliste| & \lstinline|[([Char], Integer)]|\\
  \hline
  \lstinline|(einkaufsliste, preisliste)| & \lstinline|([(Integer, [Char])], [([Char], Integer)])|\\
  \hline
\end{tabularx}

\textbf{Types of Functions and Lists}\\
Given the declarations:
\begin{lstlisting}
f1 :: Int -> Int
f1 x = x^2 + x + 1

f2 :: Int -> Int
f2 x = 2*x + 1
\end{lstlisting}
Which expressions are correctly typed?\\
\begin{tabularx}{\linewidth}{|X|X|}
  \hline
  \lstinline|f1| & \lstinline|Int -> Int|\\
  \hline
  \lstinline|f1 5| & \lstinline|Int|\\
  \hline
  \lstinline|f1 f2| & \textbf{Not correct.}\\
  \hline
  \lstinline|f1 (f2 5)| & \lstinline|Int|\\
  \hline
  \lstinline|[f1 5, f2 6, 5, 6]| & \lstinline|[Int]|\\
  \hline
  \lstinline|[f1, f2, f1]| & \lstinline|[Int -> Int]|\\
  \hline
  \lstinline|[f1 5, f2]| & \textbf{Not correct.}\\
  \hline
  \lstinline|(f1 5, f2)| & \lstinline|(Int, Int -> Int)|\\
  \hline
  \lstinline|([f1, f2, f1] !! 1) 3| & \lstinline|Int|\\
  \hline
  \lstinline|([f1, f2, f1] !! 5) 3| & \lstinline|Int|\\
  \hline
\end{tabularx}

\textbf{Types of Functions with Currying}\\
Given the declarations:
\begin{lstlisting}
g1 :: Int -> Int -> Int -> Int
g1 x y z = x^2 + y^2 + z^2

g2 :: Int -> Int -> Int
g2 x y = 2*x + 2*y
\end{lstlisting}
Which expressions are correctly typed?\\
\begin{tabularx}{\linewidth}{|l|X|}
  \hline
  \lstinline|g1| & \lstinline|Int -> Int -> Int -> Int|\\
  \hline
  \lstinline|g1 2| & \lstinline|Int -> Int -> Int|\\
  \hline
  \lstinline|g1 2 3| & \lstinline|Int -> Int|\\
  \hline
  \lstinline|g1 2 3 4| & \lstinline|Int|\\
  \hline
  \lstinline|g1 2 3 4 5| & \textbf{Not correct.}\\
  \hline
  \lstinline|(g1, g2)| & \lstinline|(Int -> Int -> Int -> Int, Int -> Int -> Int)|\\
  \hline
  \lstinline|(g1 2, g2)| & \lstinline|(Int -> Int -> Int, Int -> Int -> Int)|\\
  \hline
  \lstinline|(g1 2 3, g2 4)| & \lstinline|(Int -> Int, Int -> Int)|\\
  \hline
  \lstinline|(g1 2 3 4, g2 4 5)| & \lstinline|(Int, Int)|\\
  \hline
  \lstinline|[g1, g2]| & \textbf{Not correct.}\\
  \hline
  \lstinline|[g1 2, g2]| & \lstinline|[Int -> Int -> Int]|\\
  \hline
  \lstinline|[g1 2 3, g2 4]| & \lstinline|[Int -> Int]|\\
  \hline
  \lstinline|[g1 2 3 4, g2 4 5]| & \lstinline|[Int]|\\
  \hline
\end{tabularx}

\textbf{Polymorphic Types}\\
Given the declarations:
\begin{lstlisting}
h1 x = (x, x, x)
h2 x = [x, x, x]
h3 x = [(x, x), (x, x)]
h4 x y = (x, y)
h5 x y = [x, y]
\end{lstlisting}
Which expressions are correctly typed?\\
\begin{tabularx}{\linewidth}{|X|X|}
  \hline
  \lstinline|h1| & \lstinline|c -> (c, c, c)|\\
  \hline
  \lstinline|h2| & \lstinline|a -> [a]|\\
  \hline
  \lstinline|h3| & \lstinline|b -> [(b, b)]|\\
  \hline
  \lstinline|h4| & \lstinline|a -> b -> (a, b)|\\
  \hline
  \lstinline|h5| & \lstinline|a -> a -> [a]|\\
  \hline
  \lstinline|h1 'a'| & \lstinline|(Char, Char, Char)|\\
  \hline
  \lstinline|h1 True| & \lstinline|(Bool, Bool, Bool)|\\
  \hline
  \lstinline|h4 'a' "True"| & \lstinline|(Char, [Char])|\\
  \hline
  \lstinline|h5 'a' "True"| & \textbf{Not correct.}\\
  \hline
  \lstinline|h5 True True| & \lstinline|[Bool]|\\
  \hline
  \lstinline|[]| & \lstinline|[a]|\\
  \hline
  \lstinline|()| & \lstinline|()|\\
  \hline
  \lstinline|head []| & \lstinline|a|\\
  \hline
  \lstinline|head ()| & \textbf{Not correct.}\\
  \hline
\end{tabularx}
\end{multicols}
\textbf{Programming Exercise: two-dimensional vectors}\\
\lstinputlisting{listings/exercises/AdvPrPa_Exer01Sol_V01.hs}
\newpage
\begin{multicols}{2}

\subsubsection{Exercise 2}
\textbf{Types of numeric literals}\\
Which expressions are correctly typed?\\
\begin{tabularx}{\linewidth}{|l|X|}
  \hline
  \lstinline|2| & \lstinline|Num p => p|\\
  \hline
  \lstinline|2 + 2| & \lstinline|Num a => a|\\
  \hline
  \lstinline|2 :: Int| & \lstinline|Int|\\
  \hline
  \lstinline|2 :: Float| & \lstinline|Float|\\
  \hline
  \lstinline|(2 + 2) :: Double| & \lstinline|Double|\\
  \hline
  \lstinline|2.0| & \lstinline|Fractional p => p|\\
  \hline
  \lstinline|2.0 :: Int| & \textbf{Not correct.}\\
  \hline
  \lstinline|2 + 2.0| & \lstinline|Fractional a => a|\\
  \hline
  \lstinline|(2 :: Int) + (2 :: Double)| & \textbf{Not correct.}\\
  \hline
  \lstinline|(2 :: Int) + 2| & \lstinline|Int|\\
  \hline
  \lstinline|(2, 2)| & \lstinline|(Num a, Num b) => (a, b)|\\
  \hline
  \lstinline|[2, 2]| & \lstinline|Num a => [a]|\\
  \hline
  \lstinline|[2, 2.0]| & \lstinline|Fractional a => [a]|\\
  \hline
  \lstinline|[2 :: Float, 2 :: Double| & \textbf{Not correct.}\\
  \hline
\end{tabularx}


\textbf{Types of overloaded functions}\\
Given the declarations:
\begin{lstlisting}
f1 x = 2
f2 x = 2*x
f3 x y z = x == y && y == z
f4 x y z = x < y && y < z
f5 x y z = x == y && y < z
f6 x y = 2 * x < y
f7 x y = min (abs x) (negate y)
f8 x y = [x, y, 2]
f9 x y = x 'div' y + x / y
\end{lstlisting}
Which expressions are correctly typed?\\
\begin{tabularx}{\linewidth}{|X|X|}
  \hline
  \lstinline|f1| & \lstinline|Num p1 => p2 -> p1|\\
  \hline
  \lstinline|f1 'a'| & \lstinline|Num p1 => p1|\\
  \hline
  \lstinline|f1 "a"| & \lstinline|Num p1 => p1|\\
  \hline
  \lstinline|f1 f1| & \lstinline|Num p1 => p1|\\
  \hline
  \lstinline|f2| & \lstinline|Num a => a -> a|\\
  \hline
  \lstinline|f2 2| & \lstinline|Num a => a|\\
  \hline
  \lstinline|f2 2.0| & \lstinline|Fractional a => a|\\
  \hline
  \lstinline|f2 'a'| & \textbf{Not correct.}\\
  \hline
  \lstinline|('a', 'b') == ('c', 'd')| & \lstinline|Bool|\\
  \hline
  \lstinline|('a', 'b') < ('c', 'd')| & \lstinline|Bool|\\
  \hline
  \lstinline|('a', 'b') < ('c', 'd', 'e')| & \textbf{Not correct.}\\
  \hline
  \lstinline|['a', 'b'] < ['c', 'd', 'e']| & \lstinline|Bool|\\
  \hline
  \lstinline|f3| & \lstinline|Eq a => a -> a -> a -> Bool|\\
  \hline
  \lstinline|f3 ('a', 'b') ('a', 'b') ('a', 'b')| & \lstinline|Bool|\\
  \hline
  \lstinline|f4| & \lstinline|Ord a => a -> a -> a -> Bool|\\
  \hline
  \lstinline|f4 2 2| & \lstinline|(Ord a, Num a) => a -> Bool|\\
  \hline
  \lstinline|f5| & \lstinline|Ord a => a -> a -> a -> Bool|\\
  \hline
  \lstinline|f5 [2] [] [2,2]| & \lstinline|Bool|\\
  \hline
  \lstinline|f6| & \lstinline|(Ord a, Num a) => a -> a -> Bool|\\
  \hline
  \lstinline|(f6) 2| & \lstinline|(Ord a, Num a) => a -> Bool|\\
  \hline
  \lstinline|f7| & \lstinline|(Ord a, Num a) => a -> a -> a|\\
  \hline
  \lstinline|f7 (2 :: Int) (2 :: Integer)| & \textbf{Not correct.}\\
  \hline
  \lstinline|f8| & \lstinline|Num a => a -> a -> [a]|\\
  \hline
  \lstinline|f8 2 2.0| & \lstinline|Fractional a => [a]|\\
  \hline
  \lstinline|f9| & \lstinline|(Integral a, Fractional a) => a -> a -> a|\\
  \hline
  \lstinline|f9 2 2| & \lstinline|(Integral a, Fractional a) => a|\\
  \hline
\end{tabularx}
\end{multicols}
\textbf{Programming Exercise: List comprehensions}\\
\lstinputlisting{listings/exercises/AdvPrPa_Exer02Sol_V01.hs}
\newpage
\begin{multicols}{2}

\subsubsection{Exercise 3}
\textbf{List sugaring}\\
Rewrite the expressions so they don't contain the constrctor \lstinline|:| (cons) any longer:\\
\begin{tabularx}{\linewidth}{|X|X|}
  \hline
  \lstinline|1:2:3:[4]| & \lstinline|[1,2,3,4]|\\
  \hline
  \lstinline|1:2:[3,4]| & \lstinline|[1,2,3,4]|\\
  \hline
  \lstinline|(1:2:[]):(3:[]):[]| & \lstinline|[[1,2],3]|\\
  \hline
  \lstinline|(1,2):(3,4):[(5,6)]| & \lstinline|[(1,2),(3,4),(5,6)]|\\
  \hline
  \lstinline|[] : []| & \lstinline|[[]]|\\
  \hline
  \lstinline|[] : [] : []| & \lstinline|[[],[]]|\\
  \hline
  \lstinline|([] : []) : []| & \lstinline|[[[]]]|\\
  \hline
  \lstinline|(([] : []) : []) : []| & \lstinline|[[[[]]]]|\\
  \hline
  \lstinline|'a' : 'b' : []| & \lstinline|"ab"|\\
  \hline
\end{tabularx}

\textbf{List desugaring}\\
Rewrite the expressions so they contain the square brackets only as list constructor \lstinline|[]| (nil):\\
\begin{tabularx}{\linewidth}{|X|X|}
  \hline
  \lstinline|[1,2,3]| & \lstinline|1 : 2 : [3]|\\
  \hline
  \lstinline|[[1,2],[],[3,4]]| & \lstinline|(1:2:[]):([]):(3:4:[]):[]|\\
  \hline
  \lstinline|[[], ["a"], [[]]]| & \lstinline|???|\\ %TODO: don't know
  \hline
\end{tabularx}

\textbf{Pattern Matching}\\
Given the function and value declarations, give the type of each function and evaluate the expressions in the value declarations.\\
\begin{tabularx}{\linewidth}{|l|X|}
  \hline
  \lstinline|f1 (x : y : z) = (x, y, z)| & \lstinline|[b] -> (b, b, [b])|\\
  \hline
  \lstinline|f2 [x, y] = (x, y)| & \lstinline|[b] -> (b, b)|\\
  \hline
  \lstinline|f3 (x : y : []) = (x, y)| & \lstinline|[b] -> (b, b)|\\
  \hline
  \lstinline|a11 = f1 []| & \textbf{Not correct.}\\
  \hline
  \lstinline|a21 = f2 []| & \textbf{Not correct.}\\
  \hline
  \lstinline|a31 = f3 []| & \textbf{Not correct.}\\
  \hline
  \lstinline|a12 = f1 [1]| & \textbf{Not correct.}\\
  \hline
  \lstinline|a22 = f2 [1]| & \textbf{Not correct.}\\
  \hline
  \lstinline|a32 = f3 [1]| & \textbf{Not correct.}\\
  \hline
  \lstinline|a13 = f1 [1, 2]| & \lstinline|(1,2,[])|\\
  \hline
  \lstinline|a23 = f2 [1, 2]| & \lstinline|(1,2)|\\
  \hline
  \lstinline|a33 = f3 [1, 2]| & \lstinline|(1,2)|\\
  \hline
  \lstinline|a14 = f1 [1, 2, 3]| & \lstinline|(1,2,[3])|\\
  \hline
  \lstinline|a24 = f2 [1, 2, 3]| & \textbf{Not correct.}\\
  \hline
  \lstinline|a34 = f3 [1, 2, 3]| & \textbf{Not correct.}\\
  \hline
  \lstinline|a15 = f1 (1 : 2 : 3 : [])| & \lstinline|(1,2,[3])|\\
  \hline
  \lstinline|a25 = f2 (1 : 2 : 3 : [])| & \textbf{Not correct.}\\
  \hline
  \lstinline|a35 = f3 (1 : 2 : 3 : [])| & \textbf{Not correct.}\\
  \hline
  \lstinline|a16 = f1 ['a', 'b', 'c']| & \lstinline|('a', 'b', "c")|\\
  \hline
  \lstinline|a17 = f1 [[1], [2,3],[]]| & \lstinline|([1], [2,3], [[]])|\\
  \hline
  \lstinline|a18 = f1 (1 : 2 : 3 : [4,5])| & \lstinline|(1, 2, [3,4,5])|\\
  \hline
  \lstinline|a19 = f1 [1 .. 100]| & \lstinline|(1,2,[3,4,...,100])|\\
  \hline
\end{tabularx}
\begin{tabularx}{\linewidth}{|l|X|}
  \hline
  \lstinline|f4 (x, y) = [x, y]| & \lstinline|(a, a) -> [a]|\\
  \hline
  \lstinline|a41 = f4 ([1, 2], [3, 4, 5])| & \lstinline|[[1,2],[3,4,5]]|\\
  \hline
\end{tabularx}
\begin{tabularx}{\linewidth}{|l|X|}
  \hline
  \lstinline|g1 = "dimdi" = 1| & \\
  \hline
  \lstinline|g1 ['d', 'o', 'm', 'd', 'o'] = 2| & \\
  \hline
  \lstinline|g1 ('d' : 'i' : 'n' : 'g' : []) = 3| & \\
  \hline
  \lstinline|g1 ('d' : 'i' : 'n' : 'g' : _) = 4| & \\
  \hline
  \lstinline|g1 (x : y) = 5| & \\
  \hline
  \lstinline|g1 _ = 6| & \\
  \hline
  \lstinline|b11 = g1 "domdo"| & \lstinline|2|\\
  \hline
  \lstinline|b12 = g1 "ding"| & \lstinline|3|\\
  \hline
  \lstinline|b13 = g1 "dingdimdi"| & \lstinline|4|\\
  \hline
  \lstinline|b14 = g1 "dumdu"| & \lstinline|5|\\
  \hline
  \lstinline|b15 = g1 ""| & \lstinline|6|\\
  \hline
\end{tabularx}
\begin{tabularx}{\linewidth}{|l|X|}
  \hline
  \lstinline|g2 (d : "imdi") \| d == 'd' \|\| d == 'D' = 1| & \\%special formatting
  \hline
  \lstinline|g2 (z : "umsel") \| z == 'z' \|\| z == 'Z' = 2| & \\%special formatting
  \hline
  \lstinline|g2 _ = 3| & \\
  \hline
  \lstinline|b21 = g2 "dimdi"| & \lstinline|1|\\
  \hline
  \lstinline|b22 = g2 ['D', 'i', 'm', 'd', 'i']| & \lstinline|1|\\
  \hline
  \lstinline|b23 = g2 ('Z' : 'u' : "msel")| & \lstinline|2|\\
  \hline
  \lstinline|b24 = g2 "dimdiding"| & \lstinline|3|\\
  \hline
\end{tabularx}
\begin{tabularx}{\linewidth}{|l|X|}
  \hline
  \lstinline|h1 ['a', 'b'] = 'a'| & \\
  \hline
  \lstinline|h1 ['a', b] = b| & \\
  \hline
  \lstinline|h1 (_ : _ : 'm' : _) = 'm'| & \\
  \hline
  \lstinline|h1 (a : b) = a| & \\
  \hline
  \lstinline|c11 = h1 "ab"| & \lstinline|'a'|\\
  \hline
  \lstinline|c12 = h1 "ac"| & \lstinline|'c'|\\
  \hline
  \lstinline|c13 = h1 "dimdi"| & \lstinline|'m'|\\
  \hline
  \lstinline|c14 = h1 "zumsel"| & \lstinline|'m'|\\
  \hline
  \lstinline|c15 = h1 "schnurpsel"| & \lstinline|'s'|\\
  \hline
\end{tabularx}
\begin{tabularx}{\linewidth}{|l|X|}
  \hline
  \lstinline|h2 [(a, b), c] = c| & \\
  \hline
  \lstinline|h2 (a : b : c) = a| & \\
  \hline
  \lstinline|c21 = h2 [(1, 2), (3, 4)]| & \lstinline|(3,4)|\\
  \hline
  \lstinline|c22 = h2 [(1, 2), (3, 4), (5, 6)]| & \lstinline|(1,2)|\\
  \hline
  \lstinline|c23 = h2 [(1, 2)]| & \textbf{Not correct.}\\
  \hline
\end{tabularx}
\begin{tabularx}{\linewidth}{|X|X|}
  \hline
  \lstinline|h3 ((x : y) : z) = y| & \\
  \hline
  \lstinline|h3 ([] : _) = "2"| & \\
  \hline
  \lstinline|h3 [] = "3"| & \\
  \hline
  \lstinline|c31 = h3 ["dimdi"]| & \lstinline|"imdi"|\\
  \hline
  \lstinline|c32 = h3["", "dimdi", "domdo"]| & \lstinline|"2"|\\
  \hline
  \lstinline|c33 = h3 [[]]| & \lstinline|"2"|\\
  \hline
  \lstinline|c34 = h3 []| & \lstinline|"3"|\\
  \hline
\end{tabularx}
\end{multicols}
\textbf{Programming Exercise: Recursion over lists}\\
\lstinputlisting{listings/exercises/AdvPrPa_Exer03Sol_V01.hs}
\newpage
\begin{multicols}{2}

\subsubsection{Exercise 4}
Given the declarations, give the most general type of each value, and if the value is not a function, then evaluate it.\\
\textbf{Lambda expressions}\\
\begin{tabularx}{\linewidth}{|X|X|}
  \hline
  \lstinline|f01 :: Num a => a -> a| & \\
  \hline
  \lstinline|f01 = \\x -> 2*x| & \lstinline|Num a => a -> a|\\
  \hline
  \lstinline|f01' = \\x -> 2*x| & \lstinline|Num a => a -> a|\\
  \hline
  \lstinline|f01'' () = \\x -> 2*x| & \lstinline|Num a => () -> a -> a|\\
  \hline
  \lstinline|f01''' _ = \\x -> 2*x| & \lstinline|Num a => p -> a -> a|\\
  \hline
  \lstinline|f02 = \\x -> \\y -> x + y| & \lstinline|Num a => a -> a -> a|\\
  \hline
  \lstinline|f03 = \\x y -> x + y| & \lstinline|Num a => a -> a -> a|\\
  \hline
  \lstinline|f04 x = \\y -> x + y| & \lstinline|Num a => a -> a -> a|\\
  \hline
  \lstinline|f05 = \\(x,y) -> x + y| & \lstinline|Num a => (a,a) -> a|\\
  \hline
  \lstinline|f06 = \\[x,y] -> x + y| & \lstinline|Num a => [a] -> a|\\
  \hline
  \lstinline|f07 = [\\x -> x+1, \\x -> 2*x, \\x -> x^2]| & \lstinline|Num a => [a -> a]|\\
  \hline
  \lstinline|f08 = head f07 5| & \lstinline|Num a => a|\\
  \hline
  \lstinline|f09 = \\x -> x| & \lstinline|p -> p|\\
  \hline
  \lstinline|f10 = [f09, \\x -> x+1]| & \lstinline|Num a => [a -> a]|\\
  \hline
  \lstinline|f11 = \\_ -> (\\x -> x+1, \\() -> 'a')| & \lstinline|Num a => p -> (a -> a, () -> Char)|\\
  \hline
\end{tabularx}

\textbf{Sections}\\
\begin{tabularx}{\linewidth}{|X|X|}
  \hline
  \lstinline|x ^+^ y = x^2 + y^2| & \\
  \hline
  \lstinline|g01 = (^+^)| & \lstinline|Num a => a -> a -> a|\\
  \hline
  \lstinline|g02 = (^+^ 2)| & \lstinline|Num a => a -> a|\\
  \hline
  \lstinline|g03 = (3 ^+^)| & \lstinline|Num a => a -> a|\\
  \hline
  \lstinline|g04 = (3 ^+^ 2)| & \lstinline|Num a => a|\\
   & 13\\
  \hline
  \lstinline|g05 x y = 2*x + 3*y| & \lstinline|Num a => a -> a -> a| \\
  \hline
  \lstinline|g06 = (`g05` 2)| & \lstinline|Num a => a -> a|\\
  \hline
  \lstinline|g07 = (2 `g05`)| & \lstinline|Num a => a -> a|\\
  \hline
  \lstinline|g08 = g06 3| & \lstinline|Num a => a|\\
   & \lstinline|12|\\
  \hline
  \lstinline|g09 = g07 4| & \lstinline|Num a => a|\\
   & \lstinline|16|\\
  \hline
  \lstinline|g10 x y z = 2*x + 3*y + 4*z| & \lstinline|Num a => a -> a -> a -> a|\\
  \hline
  \lstinline|g11 = (`g05` 2)| & \lstinline|Num a => a -> a|\\
  \hline
  \lstinline|g12 = g06 3| & \lstinline|Num a => a|\\
   & \lstinline|12|\\
  \hline
  \lstinline|g13 = g07 4| & \lstinline|Num a => a|\\
   & \lstinline|16|\\
  \hline
  \lstinline|g14 x = (g10 (x+1))| & \lstinline|Num a => a -> a -> a -> a|\\
  \hline
  \lstinline|g15 = g14 2 3 4| & \lstinline|Num a => a|\\
   & \lstinline|31|\\
  \hline
  \lstinline|g16 n = \x -> ([(+), (-), (*)] !! n) x 2| & \lstinline|Num a => a -> a -> a|\\
  \hline
  \lstinline|g17 = g16 1 5| & \lstinline|Num a => a|\\
   & \lstinline|3|\\
  \hline
\end{tabularx}

\textbf{List comprehensions}\\
\begin{tabularx}{\linewidth}{|X|X|}
  \hline
  \lstinline|h01 = [x \| x <- [1 .. 5]]| & \lstinline|(Num a, Enum a) => [a]|\\
   & \lstinline|[1,2,3,4,5]|\\
  \hline
  \lstinline|h02 = [2*x \| x <- [1 .. 5]]| & \lstinline|(Num a, Enum a) => [a]|\\
  & \lstinline|[2,4,6,8,10]|\\
  \hline
  \lstinline|h03 = [x - y \| x <- [1 .. 3], y <- [1 .. 4]]| & \lstinline|(Num a, Enum a) => [a]|\\
   & \lstinline|[0,-1,-2,-3,1,0,-1,-2,|\\ %manual linebreak
   & \lstinline|2,1,0,-1]|\\
  \hline
  \lstinline|h04 = [x - y \| y <- [1 .. 3], x <- [1 .. 4]]| & \lstinline|(Num a, Enum a) => [a]|\\
   & \lstinline|[0,1,2,3,-1,0,1,2,-2,-1,|\\
   & \lstinline|0,1]|\\
  \hline
  \lstinline|h05 = [x + y \| x <- [1 .. 3], y <- [1 .. 4], x >= y]| & \lstinline|(Num a, Enum a, Ord a) => [a]|\\
   & \lstinline|[2,3,4,4,5,6]|\\
  \hline
  \lstinline|h06 = [head x \| x <- ["dimdi", "schnurpsel", "zumsel"]]| & \lstinline|[Char]|\\
   & \lstinline|"dsz"|\\
  \hline
  \lstinline|h07 = [x \| (x : _) <- ["dimdi", "schnurpsel", "zumsel"]]| & \lstinline|[Char]|\\
   & \lstinline|"dsz"|\\
  \hline
  \lstinline|h08 = [xs \| ('s' : xs) <- ["dimdi", "schnurpsel", "zumsel"]]| & \lstinline|[[Char]]|\\
   & \lstinline|["chnurpsel"]|\\
  \hline
\end{tabularx}

\end{multicols}
\subsubsection{Exercise 7}
\lstinputlisting{listings/exercises/Exercise07Solution.hs}
\textbf{Higer-Order Functions: Types}\\
\begin{tabularx}{\linewidth}{|X|X|}
  \hline
  \lstinline|f01 = curry . fst| & \lstinline|((a, b1) -> c, b2) -> a -> b1 -> c|\\
  \hline
  \lstinline|f02 = uncurry . fst| & \lstinline|(a -> b1 -> c, b2) -> (a, b1) -> c|\\
  \hline
  \lstinline|f03 = fst . curry| & \textbf{Not correct.}\\
  \hline
  \lstinline|f04 = fst . uncurry| & \textbf{Not correct.}\\
  \hline
  \lstinline|f05 = curry . curry| & \lstinline|(((a, b1), b2) -> c) -> a -> b1 -> b2 -> c|\\
  \hline
  \lstinline|f06 map ($5)| & \lstinline|Num a => [a -> b] -> [b]|\\
  \hline
  \lstinline|v07 = map ($5) [(+1), (*2)| & \lstinline|Num b => [b]|\\
   & \lstinline|[6,10]|\\
  \hline
\end{tabularx}

\textbf{Programming Exercise: Higher-order function}\\
\lstinputlisting{listings/exercises/AdvPrPa_Exer04Sol_V01.hs}
\newpage

\subsubsection{Exercise 8}
\lstinputlisting{listings/exercises/Exercise08Solution.hs}

\subsubsection{Hutton Exercises}
\textbf{Hutton02}\\
\lstinputlisting{listings/exercises/HaskellHutton02.hs}
\textbf{Hutton03}\\
\lstinputlisting{listings/exercises/HaskellHutton03.hs}
\textbf{Hutton04}\\
\lstinputlisting{listings/exercises/HaskellHutton04.hs}
\textbf{Hutton05}\\
\lstinputlisting{listings/exercises/HaskellHutton05.hs}
\textbf{Hutton06}\\
\lstinputlisting{listings/exercises/HaskellHutton06.hs}

\textbf{Programming Exercises: Higher-order functions and recursion}\\
\lstinputlisting{listings/exercises/AdvPrPa_Exer05Sol_V01.hs}

