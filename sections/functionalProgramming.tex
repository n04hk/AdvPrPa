%!TEX root = ../AdvPrPa.tex
\section{Functional Programming}
\begin{multicols}{2}
\subsection{Correctness}
(see \ref{sec:introCorrectness})

\subsubsection{Obtaining Mathematical Knowledge}
\begin{enumerate}
  \item Conjecture\\
  The product of all prime numbers between and including 2 and $p$, increased by 1, is again a prime number.
  \item Examples\\
  For $p = 2,3,5,7,11,379$ the conjecture is confirmed.
  \item Counterexample\\
  For $p=17$ the conjecture is refuted.
\end{enumerate}

\begin{enumerate}
  \item Theorem\\
  $(a+b)^2 = a^2 + 2ab + b^2$
  \item Proof\\
  $(a+b)^2 = (a+b)(a+b) = a(a+b) + b(a+b) = aa + ab + ba + bb = aa + ab + ab + bb = aa + 2ab + bb = a^2 + 2ab + b^2$
  \item[\-] with a \textbf{finite} number of steps we have thus shown that something holds for an \textbf{infinite} number of values
\end{enumerate}

\subsubsection{Consequence}
\begin{itemize}
  \item programming languages should simplify proofs (and therefore also program development itself)
  \item and thus may enhance program reliability
\end{itemize}

\subsection{Referential Transparency}
\subsubsection{A More Formal Proof}
\begin{align*}
&(a+b)^2\\
=& \{ \text{def square} \}\\
&(a+b) \cdot (a+b)\\
=& \{ \text{distri}\}\\
&a \cdot (a+b) + b \cdot (a+b)\\
=& \{ \text{distri twice}\}\\
&a \cdot a + a \cdot b + b \cdot a + b \cdot b\\
=& \{ \text{commu multi}\}\\
&a \cdot a + a \cdot b + a \cdot b + b \cdot b\\
=& \{ \text{neutral multi twice}\}\\
&a \cdot a + 1 \cdot (a \cdot b) + 1 \cdot (a \cdot b) + b \cdot b\\
=& \{ \text{distri}\}\\
&a \cdot a + (1 + 1) \cdot (a \cdot b) + b \cdot b\\
=& \{ \text{def 2}\}\\
&a \cdot a + 2 \cdot (a \cdot b) + b \cdot b\\
=& \{ \text{def square twice}\}\\
&a^2 + 2ab + b^2
\end{align*}
\begin{itemize}
  \item this proof sill handles associativity implicitly
  \item this format for \textit{calculational proofs} is due to FEIJEN and DIJKSTRA
  \item a corresponding \texttt{calc} statement is available in Dafny
\end{itemize}

\subsubsection{Equality}
A fundamental mathematical concept
\begin{itemize}
  \item four inference rules of a logic
  \item Reflexivity: $\frac{}{X=X}$
  \item Symmetry: $\frac{X=Y}{Y=X}$
  \item Transitivity: $\frac{X=Y,Y=Z}{X=Z}$
  \item LEIBNIZ: $\frac{X=Y}{E[v\leftarrow X]=E[v\leftarrow Y]}$
  \item $X,Y,Z,E$: expressions, $v$: variable, $E[v\leftarrow X]$: textual substitution of all (free) occurrences of $v$ by $(X)$ in $E$
\end{itemize}

\subsubsection{Example LEIBNIZ}
\begin{itemize}
  \item from numbers: $x \cdot (y + z) = x \cdot y + x \cdot z$
  \item therefore, by LEIBNIZ (and Substitution):
\end{itemize}
\begin{align}
  &\underbrace{(a \cdot (a + b))} + b \cdot (a + b)\\
  =& \langle \text{LEIBINIZ, with } a \cdot (a + b) = a \cdot a + a \cdot b \rangle \nonumber\\
  &\overbrace{(a \cdot a + a \cdot b)} + \underbrace{(b \cdot + (a + b))}\\
  =& \langle \text{LEIBNIZ, with } b \cdot (a + b) = b \cdot a + b \cdot b \rangle \nonumber\\
  & a \cdot a + a \cdot b + \overbrace{b \cdot a + b \cdot b}
\end{align}
\begin{itemize}
  \item therefore, since $(1) = (2)$ and $(2) = (3)$, by Transitivity: $(1) = (3)$ 
\end{itemize}

\subsubsection{Referential Transparency}
three synonymous terms
\begin{itemize}
  \item LEIBNIZ
  \item substitution of equals for equals
  \item referential transparency
\end{itemize}

\subsubsection{Functional Program}
\begin{itemize}
  \item a functional program consists of
  \begin{enumerate}
    \item a set of value and function declarations
    \item a single expression
  \end{enumerate}
  \item functional programming is referentially transparent
  \begin{itemize}
    \item values and functions are declared via equality
    \item equality then means mathematical equality (if using eager evaluation modulo termination)
  \end{itemize}
  \item referential transparency employed for
  \begin{itemize}
    \item program development, transformation, and proof
    \item evaluation
  \end{itemize}
\end{itemize}

\subsubsection{Program Transformation}
\begin{itemize}
  \item to transform a program means to rewrite it according to given rules into an equivalent program
  \item Example:
  \begin{itemize}
    \item with declaration $x = f(a)$ and arithmetic $x + x = 2 \cdot x$, expression $x + x$ can be safely rewritten into either of
    \begin{itemize}
      \item $2 \cdot x$
      \item $f(a) + x$
      \item $x + f(a)$
      \item $f(a) + f(a)$
      \item $2 \cdot f(a)$
    \end{itemize}
  \end{itemize}
\end{itemize}

\subsubsection{Evaluation}
\begin{itemize}
  \item execution of a program means evaluation of the expression
  \item Example:
  \begin{itemize}
    \item declarations: $f(x) = 2 \cdot x + 1, a = 3$
    \item expressions: $a + f(a)$
    \item evaluation:
  \end{itemize}
  \begin{align*}
    &a + f(a)\\
    =&a + (2 \cdot a + 1)\\
    =&3 + (2 \cdot 3 + 1)\\
    =&3 + (6 + 1)\\
    =&3 + 7\\
    =&10
  \end{align*}
  \item order of evaluation has no influence on result (modulo termination)
\end{itemize}

\subsection{Imperative Programming}
\begin{itemize}
  \item Example:
\begin{lstlisting}
y := 0; a := 3;
.
.
.
function f(x) begin y := y + 1; return x + y end
\end{lstlisting}
  \item execution:
  \begin{itemize}
    \item $f(a) + f(a)$ returns $4 + 5 = 9$
    \item $2 \cdot f(a)$ returns $2 \cdot 4 = 8$
  \end{itemize}
  \item no referential transparency: even the most basic arithmetic cannot be performed
  \item syntax: expressions + commands
  \item semantics: values + environment + state
  \item expressions are \textit{evaluated} in the environment and current state, yielding a value
  \item commands are \textit{executed} in the environment and current state, yielding a new state
  \item Example:
  \begin{itemize}
    \item assignment command with variable $v$ and Expression $E$
    \lstinline{v := E}
    \item $E$ is evaluated in the environment and current state, yielding value $t$; then $t$ is assigned to the storage cell denoted by $v$ in the environment, thus yielding a new state
  \end{itemize}
  \item proofs of imperative programs are well possible too, but are by far more complicated
  \item possible using HOARE logic
  \item HOARE triple, with $P$, $Q$ predicates and $C$ command
  $\{P\} C \{Q\}$
  \item means: if execution of $C$ starts in a state satisfying $P$, and execution terminates, then the resulting state satisfies $Q$
  \item Example:
  \begin{itemize}
    \item proof rule for assignment command $v := E$
    $\{Q[v \leftarrow E]\} v := E\{Q\}$
  \end{itemize}
\end{itemize}

\subsubsection{Progress in Programming Languages}
\begin{itemize}
  \item by adding features
  \begin{itemize}
    \item expressions
    \item procedures, functions
    \item types
    \item data structures
    \item abstract data types
  \end{itemize}
  \item by removing features
  \begin{itemize}
    \item gotos
    \item pointers
    \item \textbf{state and assignment}
  \end{itemize}
\end{itemize}

\subsubsection{Imperative versus Functional Programming}
\begin{itemize}
  \item imperative paradigm
  \begin{itemize}
    \item syntax: expressions + commands
    \item semantics: values + environment + state
    \item expressions are \textit{evaluated} in the environment and current state, yielding a value
    \item commands are \textit{executed} in the environment and current state, yielding a new state
  \end{itemize}
  \item functional paradigm
  \begin{itemize}
    \item syntax: expressions
    \item semantics: values + environment
    \item expressions are \textit{evaluated} in the environment, yielding a value
  \end{itemize}
\end{itemize}

\subsubsection{Misuse of the Symbol for Equality $=$}
\begin{itemize}
  \item assignment like $x := x + 1$ has not the slightest similarity to equality
  \item it is pronounced ''$x$ becomes (gets, receives) $x + 1$'' ...
  \item ... but \textbf{never ever} ''$x$ equals (is, is equal to) $x + 1$''
  \item a different symbol like $:=$ of $\leftarrow$ should be used instead
  \item using the symbol for equality $=$ to denote assignment is a horrendous design error of too many programming languages, since
  \begin{itemize}
    \item by our very basic education, it is virtually impossible to see $=$ and to not think of equality
    \item equality is such a fundamental concept that it deserves a unique non-overloaded symbol
  \end{itemize}
\end{itemize}

\subsection{Evaluation Strategies}
\subsubsection{Evaluation}
\begin{itemize}
  \item strategies
  \begin{itemize}
    \item innermost (call-by-value)
    \item outermost (call-by-name)
    \item lazy (outermost + sharing)
  \end{itemize}
  \item reducible expressions, or \textit{redex}
  \begin{itemize}
    \item application of a function to its argument expressions
  \end{itemize}
  \item Example: $\text{mult}(x, y) = x \cdot y$
  \item $\text{mult}(1 + 2, 2 + 3)$ has three redexes
  \begin{itemize}
    \item $1 + 2$, yielding $\text{mult}(3,2 + 3)$
    \item $2 + 3$, yielding $\text{mult}(1 + 2, 5)$
    \item $\text{mult}(1 + 2, 2 + 3)$, yielding $(1 + 2) \cdot (2 + 3)$
  \end{itemize}
\end{itemize}

\begin{tabularx}{\linewidth}{|X|X|}
  \hline
  \textbf{innermost} & \textbf{outermost}\\
  \hline
  innermost redex first; if several, choose leftmost one first & outermost redex first; if several, choose leftmost one first\\
  \hline
  $\text{mult}(1 + 2, 2 + 3)$ & $\text{mult}(1 + 2, 2 + 3)$\\
  $= \text{mult}(3, 2 + 3)$ & $= (1 + 2) \cdot (2 + 3)$\\
  $= \text{mult}(3, 5)$ & $= 3 \cdot (2 + 3)$\\
  $= 3 \cdot 5$ & $= 3 \cdot 5$\\
  $= 15$ & $= 15$\\
  \hline
\end{tabularx}
\begin{itemize}
  \item Example: $\text{square}(x) = x \cdot x$
  \item innermost:
  \begin{align*}
    &\text{square}(1 + 2)\\
    =&\text{square}(3)\\
    =&3 \cdot 3\\
    =&9
  \end{align*}
  \item with innermost evaluation, each argument is evaluated exactly once
  \item outermost:
  \begin{align*}
    &\text{square}(1 + 2)\\
    =&(1 + 2) \cdot (1 + 2)\\
    =&3 \cdot (1 + 2)\\
    =&3 \cdot 3\\
    =&9
  \end{align*}
  \item argument expressions might be evaluated more than once if the corresponding formal parameters occur several times in the body of the function
  \item solution to this problem via sharing:
  \begin{itemize}
    \item keep only a single copy of the argument expression, and maintain a pointer to it for each corresponding formal parameter
    \item evaluate the expression once, and replace it by its value
    \item access this value through the pointers
  \end{itemize}
\end{itemize}

\subsubsection{Evaluation}
\begin{itemize}
  \item Example:
  \begin{enumerate}
    \item $f(x) = 17$
    \item $\text{inf}(x) = \text{inf}(x)$
  \end{enumerate}
  \item $\text{inf}(0)$ obviously yields an endless recursion
  \item What is $f(\text{inf}(0))$?
  \item What is $f(1 \text{div} 0)$?
  \item innermost:
  \begin{itemize}
    \item $f(\text{inf}(0))$ yields an endless recursion
    \item $f(1 \text{div} 0)$ aborts
  \end{itemize}
  \item outermost (and thus lazy):
  \begin{itemize}
    \item $f(\text{inf}(0))$ yields $17$
    \item $f(1 \text{div} 0)$ yields $17$
  \end{itemize}
  \item an argmument is evaluated
  \begin{itemize}
    \item innermost: exactly once
    \item outermost: zero or more times
    \item lazy: at most once
  \end{itemize}
  \item whenever there exists an order of evaluation that terminates, outermost (and thus lazy) evaluation will find it
\end{itemize}




\end{multicols}
